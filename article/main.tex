\documentclass{article}

\usepackage{hyperref}
\usepackage{enumerate}
\usepackage{enumitem}
\usepackage{amssymb,amsmath,amsthm} %must be before unicode-math
\usepackage[mathletters]{ucs}
\DeclareUnicodeCharacter{8348}{_t}
\DeclareUnicodeCharacter{8339}{_x}
\DeclareUnicodeCharacter{8345}{_n}
\DeclareUnicodeCharacter{7522}{_i}
\DeclareUnicodeCharacter{11388}{_j}
\DeclareUnicodeCharacter{7523}{_r}
\usepackage[utf8x]{inputenc}
\usepackage{bussproofs}
\usepackage{graphicx}
\usepackage{stmaryrd}
\usepackage[autosize]{dot2texi}
\usepackage{multicol}
\setlength{\multicolsep}{0pt}
\usepackage{tikz}
\usetikzlibrary{shapes,arrows,tikzmark,decorations.pathreplacing,calc}
\usepackage{adjustbox}

% Title portion. Note the short title for running heads
\title{Solving smooth system of algebraic equations}
\author{Christophe Raffalli}

\newcommand{\interior}[1]{%
  #1^{\mathrm{o}}%
}
\newcommand{\cardinal}[1]{%
  \#{#1}%
}
\newcommand{\PNR}{{\cal P}^n(ℝ)}
\newcommand{\SNR}{{\cal S}^n(ℝ)}

\newtheorem{theo}{Theorem}
\newtheorem{nota}[theo]{Notation}
\newtheorem{defi}[theo]{Definition}
\newtheorem{prop}[theo]{Proposition}

% Document starts
\begin{document}


\maketitle

\begin{nota}
  We consider the real projective plane $\PNR$.
  When ${x ∈ ℝ^{n+1}}$, we write $\overline{x} = \{ λx, λ∈ℝ_* \} ∈
  \PNR$ the equivalent class of $x$. We define $πₙ:ℝ^{n+1} → \PNR$ the
  projection $x ↦ \overline{x}$.
  We use on $\PNR$ the measure induced by
  the chord metric of the unit sphere $\SNR$:
  $$d(\overline{x},\overline{y}) = \min\left(
  \left\|{x \over \|x\|} - {y \over \|y\|}\right\|, \left\|{x \over \|x\|} + {y
    \over \|y\|}\right\|\right) (∀\overline{x},\overline{y} ∈ \PNR).$$

  We write $\cardinal{S}$ the cardinal of a set $S$.
\end{nota}

\begin{defi}
  A \emph{simplicial partition} of
  $\PNR$ is a finite set of closed simplices $(Δᵢ)_{i∈I}$ such that
  \begin{itemize}
  \item $⋃_{i∈I} Δᵢ = \PNR$,
  \item for all $i≠j ∈ I$, $Δᵢ ∩ Δⱼ$ is a included in an hyperplane of $\PNR$.
  \item for all simplices in $Δ$, there is a projective plane of $\PNR$ that do
    not meet $Δ$.
  \end{itemize}
  We say that $Δᵢ$ and $Δⱼ$ are neighbour if $Δᵢ ∩ Δⱼ$ is infinite,
  i.e. contains at least two points. We do not consider simplices sharing only
  on vertex to be neighbour.  The neighbourhood relation is clearly reflexive
  and symmetric.  For $i \in I$, we define $V(i) = \{ j \in I; \cardinal{(Δᵢ ∩
    Δⱼ)} > 2\}$.
\end{defi}

\begin{defi}
  Let $ε$ be a positive real. A \emph{simplicial partition} $(Δᵢ)_{i∈I}$ of $\PNR$ is said to be
  $ε$-fat if for any $i,j ∈ I$, for any $x∈Δᵢ$ and $y ∈ Δⱼ$, if $d(x,y) <  ε$,
  then $Δᵢ ∩ Δⱼ ≠ ∅$.
\end{defi}

\begin{prop}
  Any \emph{simplicial partition} $(Δᵢ)_{i∈I}$ of $\PNR$ is $ε$-fat for
  some $ε > 0$ small enough.
\end{prop}

\begin{proof}
  Take $ε = \displaystyle{\min_{i,j ∈ I, Δᵢ ∩ Δⱼ = ∅} d(Δᵢ,Δⱼ)}$ which is positive because $I$ is
  finite and for all $i ∈ I$, $Δᵢ$ is compact.
\end{proof}

\begin{defi}
   Let $Δ = (Δᵢ)_{i∈I}$ be  a \emph{simplicial partition} of $\PNR$, a
   \emph{$Δ$-piecewise linear map}
   is a map $f : ℝ^{n+1} → ℝ$ which is
continuous, piecewise linear and linear
on each $πₙ^{-1}(Δᵢ)$ for $i ∈ I$.

  For such a map, an for all $x ∈ π^{-1}Δᵢ$, we define $∇ᵢf(x) = ∇ᵢf = ∇f(y)$ for any $y ∈
  π^{-1}\interior{Δᵢ}$. For
  any point $x∈ℝ^{n+1}$, we define $∇_cf(x)$ the convex-hull of $\{∇ᵢf(x),
  \hbox{ for } i \hbox{ s.t. } x ∈
  π^{-1}Δᵢ\}$. For $x$ in $π^{-1}\interior{Δᵢ}$, $∇_cf(x) = \{∇f(x)\}$.
\end{defi}

\begin{defi}
  We say that a $Δ$-map $f$ is smooth if for all $x ∈ \PNR$, $∇_cf(x)$ does not
    contains $0$.
\end{defi}

\begin{prop}
  Let $Δ = (Δᵢ)_{i∈I}$ be a \emph{simplicial partition} of $\PNR$ and
  let $f$ be a smooth \emph{$Δ$-map}.
  For any $ε_M>0$ such that $Δ$ is $ε$-fat, there exists a family of $C^∞$ function
  $ε ↦ f_ε : \SNR → ℝ$ such that
  \begin{itemize}
  \item $∀x ∈ \SNR, f(x) - f_ε(x) → 0$ when $ε → 0$
  \item $∀i ∈ I, ∀x ∈ ℝ^{n+1}, ∇f_ε(x) ∈ ∇_cf(x)$.
  \item All functions $f_ε$ and $f$ have homeomorphic zero level.
  \end{itemize}
\end{prop}

\begin{proof}
  Let $B_r(x)$ denote the unit ball of center $x$ and radius $r$.
  Let $s:ℝ^{n+1} → ℝ⁺$ a $C^∞$ map with support included in $B_1(x)$ and such
  that $∫_{ℝ^{n+1}} s(t)dt = 1$ for the usual measure $dt$ of $ℝ^{n+1}$. We also
  assume that $s(-x) = s(x)$.
  We define $s_ε(x) = ε^{-n-1}s(ε^{-1} x)$, which has support in $B_ε(x)$ and
  satisfies $∫_{ℝ^{n+1}} s_\epsilon(t)dt = 1$

  We define $f_ε(x) = f ∗ s = ∫_{ℝ^{n+1}} f(x-t)s_ε(t) dt$ the convolution product of $f$ and
  $s_{ε}$. $f_ε$ is $C^∞$ on $\SNR$. $|f(x) - f_ε(x)| = |∫_{ℝ^{n+1}} (f(x) - f(t))s_{ε}(x-t) dt| <
  \max_{t∈B_ε(x)} |f(x) - f(t)| → 0$  when $ε → 0$ because $f$ is continuous.

  Remark: $f_ε$ is not homogeneous, this is why we restrict its domain to
  $\SNR$.

  For the gradient, we have,
  \begin{eqnarray*}
    ∇f_ε(x) &=& ∇∫_{ℝ^{n+1}} f(x-t)s_{ε}(t) dt\cr
     &=& ∇∫_{ℝ^{n+1}} f(u)∇s_{ε}(x-u) du\cr
     &=& ∫_{ℝ^{n+1}} f(u)∇s_{ε}(x-u) du\cr
    &=& Σ_{i∈V(x)}∫_{Δᵢ} f(x-t)∇s_{ε}(t) dt\cr
    &=& Σ_{i∈v(x)}∫_{Δᵢ} ∇f(x-t)s_{ε}(t) dt\cr
    &=& Σ_{i∈V(x)} λᵢ ∇ᵢf \hbox{ with } λᵢ = ∫_{Δᵢ} s_{ε}(x-t) dx > 0
  \end{eqnarray*}

  Hence $∇f_ε(x)$ is in the convex hull of the $∇ᵢf$ for $i ∈ V(x)$ because
  $$Σ_{i∈I(x)} λᵢ = Σ_{i∈v(x)}∫_{Δᵢ} s_ε(x-t) dx =  1$$

  To build the homeomorphism between the variaous zero levels, we consider the
  following differential equation on for $γ(ε) : ℝ ↦ ℝ^{n+1}$:

  $$ γ'(ε) = g(ε,γ(ε)) \hbox{ with } g(ε,x) = \frac{\frac{∂f_ε}{∂ε}(x) ∇f_ε(x)}{||∇f_ε(x)||} $$

  The function $g$ is ${\cal
    C}^∞$ for $0 < ε < ε_M$ and $x ∈ ℝ^{n+1}$ because $∇f_ε(x) ≠ 0$ on the
  considered domain. the first thing to show it can be
  prolongated to $ε = 0$. We have

  \begin{eqnarray*}
    ∇s_ε(x) &=& ∇(ε\|x\|)^{-n-1} s((ε\|x\|)^{-1}x) \cr
                    &=& (-n-1) (ε\|x\|)^{-n-2} s((ε\|x\|)^{-1}x)
    }\frac{x}{\|x\|} \cr
     && + (ε\|x\|)^{-n-1} ∇s((ε\|x\|)^{-1}x) \left((ε\|x\|)^{-1} -
      (ε\|x\|)^{-2}  \frac{x}{\|x\|}\right) \cr
     && + (ε\|x\|)^{-n-1} ∇s((ε\|x\|)^{-1}x) \left((ε\|x\|)^{-1} -
      (ε\|x\|)^{-2}\right)  \frac{x}{\|x\|} \cr
  \end{eqnarray*}
  \begin{eqnarray*}
    \frac{∂}{∂ε} s_{ε\|x\|}(x-t)
      &=& \frac{∂}{∂ε} (ε\|x\|)^{-n-1} s((ε\|x\|)^{-1}(x-t)) \cr
    &=& \frac{∂}{∂ε} (-n-1)\|x\|(ε\|x\|)^{-n-2} s((ε\|x\|)^{-1}(x-t))
                      - ε^{-1}(ε\|x\|)^{-n-2} ∇s((ε\|x\|)^{-1}(x-t)).(x-t)\cr
    &=& \frac{∂}{∂ε} (-n-1)ε s_ε(x-t)
                      - ε^{-1}(ε\|x\|)^{-n-2} ∇s((ε\|x\|)^{-1}(x-t)).(x-t)\cr
  \end{eqnarray*}

    \begin{eqnarray*}
    \frac{∂f_ε(x)}{∂ε} &=& \frac{∂}{∂ε} ∫_{ℝ^{n+1}} f(x-t)s_{ε}(t) dt \cr
    &=& \frac{∂}{∂ε} ∫_{ℝ^{n+1}} f(t)s_{ε\|x\|}(x-t) dt \cr
    &=&  \frac{∂}{∂ε} Σ_{i∈V(x)} ∫_{Δᵢ} f(t)s_{ε\|x\|}(x-t) du \cr
    &=&  Σ_{i∈V(x)} ∫_{Δᵢ} f(t) \frac{∂}{∂ε} s_{ε\|x\|}(x-t) du \cr
       &=& Σ_{i∈V(x)}∫_{Δᵢ} f(x-t) s_{ε\|x\|}(t) dt
  \end{eqnarray*}



\end{proof}

  \begin{prop}
    Let $p$ be an homogeneous polynomial in $n+1$ variables.
    Let $Δ$ be an ad
  \end{prop}
  \begin{proof}

  \end{proof}
\end{document}
