\documentclass{article}

\usepackage{hyperref}
\usepackage{enumerate}
\usepackage{enumitem}
\usepackage{amssymb,amsmath,amsthm} %must be before unicode-math
\usepackage[mathletters]{ucs}
\DeclareUnicodeCharacter{8348}{_t}
\DeclareUnicodeCharacter{8342}{_k}
\DeclareUnicodeCharacter{8339}{_x}
\DeclareUnicodeCharacter{8345}{_n}
\DeclareUnicodeCharacter{7522}{_i}
\DeclareUnicodeCharacter{11388}{_j}
\DeclareUnicodeCharacter{7523}{_r}
\DeclareUnicodeCharacter{7580}{^c}
\DeclareUnicodeCharacter{7496}{^d}
\usepackage[utf8x]{inputenc}
\usepackage{graphicx}
\usepackage{stmaryrd}
\usepackage[autosize]{dot2texi}
\usepackage{multicol}
\setlength{\multicolsep}{0pt}
\usepackage{tikz}
\usetikzlibrary{shapes,arrows,tikzmark,decorations.pathreplacing,calc}
\usepackage{adjustbox}

% Title portion. Note the short title for running heads
\title{Solving smooth system of algebraic equations}
\author{Christophe Raffalli}

\newcommand{\interior}[1]{%
  #1^{\mathrm{o}}%
}
\newcommand{\cardinal}[1]{%
  \#({#1})%
}
\newcommand{\hull}[1]{%
  \mathcal H({#1})%
}
\newcommand{\cone}[1]{%
  \mathcal C({#1})%
}
\newcommand{\vertices}[1]{%
  \mathcal V({#1})%
}
\newcommand{\ball}[2]{%
  \mathcal B_{#2}({#1})%
}

\newcommand{\PNR}{{\cal P}^n(ℝ)}
\newcommand{\SNR}{{\cal S}^n(ℝ)}

\newtheorem{theo}{Theorem}
\newtheorem{nota}[theo]{Notation}
\newtheorem{defi}[theo]{Definition}
\newtheorem{prop}[theo]{Proposition}
\newtheorem{exam}[theo]{Example}

% Document starts
\begin{document}


\maketitle

\section{Introduction}

\section{Conventions and notations}

We consider the real projective plane $\PNR$.
 We define $πₙ:ℝ^{n+1}∖\{0\} → \PNR$ the
projection $x ↦  \{ λx, λ∈ℝ^⋆ \} ∈ \PNR$. To simplify, we often write $\overline{x}$ for $πₙ(x)$.


We use on $\PNR$ and $\SNR$ the measure induced by
the chord metric of the unit sphere:
$$d(\overline{x},\overline{y}) = \min\left(
\left\|{x \over \|x\|} - {y \over \|y\|}\right\|, \left\|{x \over \|x\|} + {y
  \over \|y\|}\right\|\right) (∀\overline{x},\overline{y} ∈ \PNR).$$

If $x ∈ \PNR$ (resp. $ℝ^{n+1}$), $\ball{x}{r}$ denotes the closed ball of center $x$
and radius $r$.

We write $\cardinal{S}$ the cardinal of a set $S$.

For $A ⊂ ℝ^{n+1}$, $\hull{A}$ denotes the convex hull of $A$ and $\cone{A}$ its
convex cone.

We use the following notation for simplices:
\begin{itemize}
\item If $Δ ⊂ \PNR$ is a simplex, we denote $\vertices{Δ} ⊂ \SNR$ a finite set of
  represents of the vertices of $Δ$.
\item $\vertices{Δ}$ must verify that $Δ = πₙ(\hull{\vertices{Δ}})$. This is why we take $\vertices{Δ} ⊂
    \SNR$ instead of $\vertices{Δ} ⊂ \PNR$ as the convex hull is not well defined in
    the projective plane.
\item Simplices should not be too big, in particular, $\PNR$ is not a simplex!
  For this we require $\vertices{Δ}$ to fit in an open half space of
  $\SNR$. If this is not possible, we consider that $Δ$ is not a simplex.
\item We take $\vertices{Δ}$ to be minimal and consider only non degenerated simplex: removing one point in
    $\vertices{Δ}$
    changes it convex hull and the dimension of the
    simplex $Δ$ is always exactly $\cardinal{\vertices{Δ} - 1}$. This also means that
    $\vertices{Δ}$ is the set of extremal points of its convex hull.
  \item Because we require simplices to be not too big $πₙ^{-1}(Δ)$ has exactly two connected components:
    $$πₙ^{-1}(Δ) = \cone{\vertices{Δ}} ∪ -\cone{\vertices{Δ}}.$$
    We write $Δ⁺ =  \cone{\vertices{Δ}}$ and $Δ⁻ = - Δ⁺$. Here $-A$ denotes
    $\{-x, x ∈ A\}$.

\end{itemize}

\section{Simplicial partition}

\begin{defi}
  A \emph{simplicial partition} of
  $\PNR$ is a finite set of closed simplices $(Δᵢ)_{i∈I}$ such that
  \begin{itemize}
  \item Each $Δᵢ$ is a simplex of dimension $n$, i.e. $\cardinal{\vertices{Δ}} = n + 1$.
  \item $⋃_{i∈I} Δᵢ = \PNR$,
  \item For all $i₁ < i₂ < \dots iₖ ∈ I$, $Δ_{i₁} ∩ \dots ∩ Δ_{iₖ}$ is empty or is a simplex of dimension at most
    $n-k+1$ and $πₙ(\vertices{Δ_{i₁} ∩ \dots ∩ Δ_{iₖ}}) =
    πₙ(\vertices{Δ_{i₁}}) ∩ \dots ∩ πₙ(\vertices{Δ_{iₖ}})$.
    We can not require  $\vertices{Δᵢ ∩ Δⱼ} = \vertices{Δᵢ} ∩ \vertices{Δⱼ}$ because we may have $x ∈
    \vertices{Δᵢ}$ and $y ∈ \vertices{Δⱼ}$ with $x ≠ y$ and $\overline{x} = \overline{y}$ (see
    the example below). This conditions ensures that the intersection of
    simplices are exactly faces of lower dimensions defined by the common vertices.
%  \item We say that $Δᵢ$ and $Δⱼ$ are neighbour if $Δᵢ ∩ Δⱼ ≠ ∅$. The neighbourhood relation is clearly reflexive and symmetric.
  \end{itemize}
\end{defi}

Here is as example a partition of $\PNR$ with $2ⁿ$ simplices:
\begin{exam}\label{init_part}
Consider $B = \{x₀,\dots,xₙ\}$, the canonical base of $ℝ^{n+1}$ and
$(εᵢ)_{i ∈ \{1,\dots,2ⁿ\}}$ an enumeration of all sequences of length $n$ of
$1$ or $-1$. Then, we define $(Δᵢ)_{i∈\{1,\dots,2ⁿ\}}$ by
$$Δᵢ = πₙ(\hull{\{x₀,ε_{i,1} x₁, \dots, ε_{i,n} xₙ\}}).$$
\end{exam}

We remark in this example, that all simplices use $B$ as set of vertices. This
means that for all $i < j ∈ \{1,\dots,2ⁿ\}$, we have
  $πₙ(\vertices{Δᵢ}) = πₙ(\vertices{Δⱼ}) = πₙ(B)$ while $\vertices{Δᵢ} ≠ \vertices{Δⱼ}$.

In this examples, we have
\begin{eqnarray*}
  Δ⁺ᵢ &=& \cone{\{x₀,ε_{i,1} x₁, \dots, ε_{i,n} xₙ\}} \cr
  Δ⁻ᵢ &=& \cone{\{-x₀,-ε_{i,1} x₁, \dots, -ε_{i,n} xₙ\}} \cr
\end{eqnarray*}

\begin{defi} Let $(Δᵢ)_{i∈I}$ be a simplicial partition of $\PNR$. For $x ∈
  ℝ^{n+1} ∖\{0\}$ we define
  $I^Δ(x) = \{(i,σ) ∈ I × \{-,+\}, x ∈ Δᵢ^σ\}$ and
  $I^Δ_ε(x) = \{(i,σ) ∈ I × \{-,+\}, Δᵢ^σ ∩ \ball{x}{ε\|x\|} ≠ ∅\}$. In general we only use one
  simplicial partition at a time and we simply write $I(x)$ and $I_ε(x)$.
\end{defi}

\begin{prop}\label{Ihomo}
  Let $(Δᵢ)_{i∈I}$ be a simplicial partition of $\PNR$.
  For $x ∈  ℝ^{n+1} ∖\{0\}$ and $λ > 0$, we have $I(λx) = I(x)$ and $I_ε(λx) = I_ε(x)$.
\end{prop}

\begin{proof}
  From the fact that the set $Δᵢ^σ$ are convex cones.
\end{proof}

\begin{prop}
  Let $(Δᵢ)_{i∈I}$ be a simplicial partition of $\PNR$.
  For any $x ∈ ℝ^{n+1} ∖\{0\}$, if $ε>0$ is small enough, then $I_ε(x) = I(x)$.
\end{prop}

\begin{proof}
  Take $ε < \min_{(i,σ) ∉ I(x)} d(x,Δᵢ^σ)$, then we have $I_ε(x) = I(x)$.
\end{proof}

\begin{defi}
  Let $ε$ be a positive real. A \emph{simplicial partition} $(Δᵢ)_{i∈I}$ of
  $\PNR$ is said to be $ε$-fat if for all $x ∈ ℝ^{n+1} ∖\{0\}$ we have:
$$ ⋂_{(i,σ) ∈ I_ε(x)} Δᵢ^σ ≠ ∅$$
\end{defi}

\begin{prop} If a simplicial partition $(Δᵢ)_{i∈I}$ is $ε$-fat for $ε > 0$
  then it is $η$-fat if $0 < η ≤ ε$.
\end{prop}

\begin{proof} From the fact that for $x ∈ ℝ^{n+1} ∖\{0\}$, $I_η(x) ⊆ I_ε(x)$ if $η ≤ ε$.
\end{proof}

\begin{prop}
  Any \emph{simplicial partition} $(Δᵢ)_{i∈I}$ of $\PNR$ is $ε$-fat for
  some $ε > 0$ small enough.
\end{prop}

\begin{proof}
  First, thanks to proposition \ref{Ihomo}, we only have to check the condition
  for $ε$-fatness in $\SNR$.

  Consider a set $J ⊂ I × \{-,+\}$ with $⋂_{(i,σ)∈J} Δⱼ^σ = ∅$ (1), we first
  prove that there exists $ε_J > 0$ such that for all $x ∈ \SNR$, $J \not⊆
  I_{ε_J}(x)$. For this we assume the contrary: For any $ε>0$ there exists $x_ε
  ∈ \SNR$ such that $J ⊆ I_{ε}(x_ε)$. As $\SNR$ is compact, we can extract a
  sequence $(εₙ,xₙ)_{n ∈ \mathbb N}$ which converges to $(0,x)$ for some $x ∈
  \SNR$, with $J ⊆ I_{εₙ}(x_{εₙ})$ for all $n ∈ \mathbb N$. Thus, for any $(i,σ)
  ∈ J$ and $n ∈ \mathbb N$, we have $\ball{xₙ}{εₙ} ∩ Δᵢ^σ ≠ ∅$ which implies
  $d(xₙ,Δᵢ^σ) < εₙ$. As $Δᵢ^σ$ are closed convex cone, this implies that $x ∈
  Δᵢ^σ$ for any $(i,σ) ∈ J$ which contradicts the hypothesis (1).

  We end the proof by remarking that the simplicial partition is $ε$-fat if we
  take $ε$ the minimum of all $ε_J$ for $J ⊂ I$ with $∩_{(i,σ)∈J} Δᵢ^σ =
  ∅$. This is well defined as $I$ hence his power-set are finite. Indeed, if $J
  = I_{ε}(x)$, and $∩_{(i,σ)∈J} Δᵢ^σ = ∅$, then $J \not ⊆ I_{ε_J}(x) ⊇ I_{ε}(x)
  = J$ which is contradictory.
\end{proof}

\section{Δ-linear approximation of a polynomial}

\begin{defi}
Let $p$ be an homogeneous polynomial of degree $d$ on
$ℝ^{n+1}$. Let $(Δᵢ)_{i∈I}$ be a simplicial partition of $\PNR$.
We define $\overline{p} : ℝ^{n+1} → ℝ$, the \emph{$Δ$-linear approximation of $p$},
the piecewice linear function
that is linear on all $Δ^σᵢ$ for $i ∈ I$, $σ ∈ \{+,-\}$, and equal to $p$
on all points in $\vertices{Δ}$.

For $i ∈ I$, $σ ∈ \{+,-\}$, we define $∇ᵢ^σp = ∇\overline{p}(x)$
for any $x ∈ Δ^σᵢ$, as the gradient is constant over $Δ^σᵢ$ for a linear function.
\end{defi}

Remark: if $d$ is even (resp. odd), $\overline{p}$ is even (resp. odd).

\begin{defi}
Let $p$ be an homogeneous polynomial of degree $d$ on
$ℝ^{n+1}$. Let $(Δᵢ)_{i∈I}$ be an $ε$-fat simplicial partition of $\PNR$.
Let $\overline{p}$ be the $Δ$-linear approximation of $p$.
We define the condition $Π_ε(x)$ to be true if and only if $p$ and
$\overline{p}$ are both positive or both negative on $\ball{x}{ε}$.
\end{defi}

\begin{prop}\label{adaptedeq}
Let $p$ be an homogeneous polynomial of degree $d$ on
$ℝ^{n+1}$. Let $(Δᵢ)_{i∈I}$ be an $ε$-fat simplicial partition of $\PNR$.
Let $\overline{p}$ be the $Δ$-linear approximation of $p$.

The following conditions are equivalent:
\begin{enumerate}
\item $∀x ∈  ℝ^{n+1}∖\{0\}, Π_ε(x) \hbox{ or } 0 ∉ \cone{\{∇p(x)\} ∪ \{∇ᵢ^σp, (i,σ) ∈ I_ε(x)\}}$
\item $∀x ∈  ℝ^{n+1}∖\{0\}, Π_ε(x) \hbox{ or } 0 ∉ \hull{\{∇p(x)\} ∪ \{∇ᵢ^σp, (i,σ) ∈ I_ε(x)\}}$
\item $∀x ∈  \SNR, Π_ε(x) \hbox{ or } 0 ∉ \cone{\{∇p(x)\} ∪ \{∇ᵢ^σp, (i,σ) ∈ I_ε(x)\}}$
\item $∀x ∈  \SNR, Π_ε(x) \hbox{ or } 0 ∉ \hull{\{∇p(x)\} ∪ \{∇ᵢ^σp, (i,σ) ∈ I_ε(x)\}}$
\item\label{hb} $∀x ∈  ℝ^{n+1}∖\{0\}, Π_ε(x) \hbox{ or } ∃d ∈ ℝ^{n+1}∖\{0\},
  \left\{\begin{array}{l} d.∇p(x) > 0 \hbox{ and}\cr
    ∀(i,σ) ∈ I_ε(x), d.∇ᵢ^σp > 0
  \end{array}\right.$
\end{enumerate}
\end{prop}

\begin{proof}
  The equivalences (1) ↔ (2) and (3) ↔ (4) are a general property of convex hull
  and convex cone: for any set $X ⊂ ℝ^{n+1}$, $0 ∈ \hull{X} ↔ 0 ∈ \cone{X}$.
  %Left to right is because $\hull{X} ⊆ \cone{X}$. For right to left, if $0 ∈
  %\cone{X}$, if means we write $0 = λᵢ x₁ + \dots + λₙ xₙ$ with $λᵢ > 0$ and
  %$xᵢ ∈ X$ for $1 < i < n$. Hence, taking $μᵢ = \frac{λᵢ}{λᵢ + \dots + λₙ}$, we
  %have $0 = μᵢ x₁ + \dots + μₙ xₙ$ with $μ₁ + \dots + μₙ = 1$ and $μᵢ > 0$, $xᵢ
  %∈ X$ for $1 < i < n$.


  The equivalence (1)  ↔ (3) comes from the fact that $p$ is homogeneous hence
  $\nabla(p)(λx) = λ^{d-1}\nabla(p)(x)$ for $λ > 0$ and from proposition $\ref{Ihomo}$.

  The equivalence (2) ↔ (5) is a form of Han-Banach theorem \cite{?}.
\end{proof}

\begin{defi}
Let $p$ be an homogeneous polynomial of degree $d$ on
$ℝ^{n+1}$. Let $(Δᵢ)_{i∈I}$ be an $ε$-fat simplicial partition of $\PNR$.

We say that the partition is \emph{adapted} to $p$ if it satisfies the condition
of proposition \ref{adaptedeq}.
\end{defi}

\begin{prop}
  Let $p$ be an homogeneous polynomial of degree $d$ on
$ℝ^{n+1}$.
  Let $(Δᵢ)_{i∈I}$ be an $ε$-fat simplicial partition of $\PNR$ adapted to $p$.
  We can find a $C^∞$ function $d$ in $ℝ^{n+1} ∖ \{0\} → ℝ^{n+1}∖\{0\} $
  and $μ > 0$ such that:
$$∀x ∈  ℝ^{n+1}∖\{0\}, d(x).∇p(x) > μ\|x\|^{d-1} \hbox{ and }∀(i,σ) ∈
  I_ε(x), d(x).∇ᵢ^σp > μ.$$
\end{prop}

\begin{proof}
From proposition \ref{adaptedeq}.\ref{hb}, for any $x ∈ ℝ^{n+1}∖\{0\}$, we find $d₀(x)
∈ ℝ^{n+1}$ such that $d₀(x).∇p(x) > 0$ and $∀(i,σ) ∈ I_ε(x), d₀(x).∇ᵢ^σp > 0.$.
We can also assume that $d₀(λ x) = d₀(x)$ for $λ > 0$ for the same reason as in
the proof of $(1) ↔ (3)$ above.

We start with the following lemma: for each $x ∈
ℝ^{n+1}∖\{0\}$, we find $0 < η(x) < ε$ such that
$∀y ∈ \ball{x}{2η(x)}$, $d₀(x).∇p(y) > 0$ and $∀(i,σ) ∈ I_ε(y),
d₀(x).∇ᵢ^σp > 0.$.

First, $(i,σ) ∈ I_ε(y) ↔ \ball{y}{ε} ∩ Δᵢ^σ ≠ ∅$ is a closed condition and the
contrary is an open condition. Hence, for $x ∈ℝ^{n+1}∖\{0\}$, we find $η > 0$
small enough to have $y ∈
\ball{x}{2η}$ gives $(i,σ) ∉ I_ε(x)$ implies $(i,σ) ∉ I_ε(y)$, which means that
for $y ∈ \ball{x}{2η}$, we have $I_ε(y) ⊆ I_ε(x)$ and therefore,

$$∀y ∈ \ball{x}{2η(x)}, ∀(i,σ) ∈ I_ε(y), d₀(x).∇ᵢ^σp > 0.$$

Second, $∇p(x)$ is continuous, thus for  $x ∈ℝ^{n+1}∖\{0\}$, we find $η > 0$
small enough to have $y ∈ \ball{x}{2η}$
implies $d₀(x).∇p(y) > 0$. Grouping both results yields the wanted real $η(x)$
such that

$$∀x ∈ℝ^{n+1}∖\{0\}, ∀y ∈ \ball{x}{2η(x)},
\left\{\begin{array}{l} d₀(x).∇p(y) > 0 \hbox{ and}\cr
∀(i,σ) ∈ I_ε(y), d₀(x).∇ᵢ^σp > 0
\end{array}\right.
$$


Now, for $x ∈ \SNR$, the balls $\ball{x}{η(x)}$ cover $\SNR$, so we can
find a finite family $(xⱼ)_{j∈J}$ such that the balls $\ball{x}{η(xⱼ)}$ cover
$\SNR$. We also take $η = min_{j ∈ J} η(xⱼ)$.

Nest, for each $xⱼ ∈ \SNR$, $j ∈ J$, we choose a measurable set
$Dⱼ ⊂ \ball{x}{η(xⱼ)}$ such that $\SNR$ is a disjoin union of all
$Dⱼ$. And we define $d₁ : \SNR → ℝ^{n+1}∖\{0\}$ with $d₁(x) = d₀(xⱼ)$ if $x ∈ Dⱼ$.

We choose $s : ℝ^{n+1} → ℝ$, a $C^∞$ function with support in
$\ball{0}{1}$ and $∫_{ℝ^{n+1}} s(t) dt = 1$ for the standard measure $dt$
on $ℝ^{n+1}$.

We define

$$ d : \SNR → ℝ^{n+1}∖\{0\} \hbox{ with } d(x) = ∫_{ℝ^{n+1}} d₁(x - η t)
s(t) dt$$

Similarly to a convolution product, we have $d(x) = ∫_{ℝ^{n+1}} η^{n +1} d₁(u)
s(\frac{x - u}{η}) du$ and therefore $d$ is $C^∞$ on $\SNR$.

Moreover, if we take $x ∈ \SNR$, we can write
$$ d(x) = Σ_{j∈J} d₁(xⱼ) ∫_{x - η t ∈ Dⱼ} s(t) dt$$

Let us define $λⱼ(x) = ∫_{x - η t ∈ Dⱼ} s(t) dt$. As $Dⱼ ⊂ \ball{xⱼ}{η(xⱼ)}$,
if $λⱼ(x) > 0$, we have $d(x - η t,xⱼ) ≤ η(xⱼ)$ otherwise $x  - η t ∉ Dⱼ$
and $d(x, x - η t) < η ≤ η(xⱼ)$ otherwise $s(t) = 0$. Therefore,
if $λⱼ(x) > 0$, we have $d(x,xⱼ) < 2η(xⱼ)$.
which implies $d₁(xⱼ).∇p(x) > 0$ and $∀(i,σ) ∈ I_ε(x), d₁(xⱼ).∇ᵢ^σp > 0$.

Moreover, the coefficient $λⱼ$ are positive. Therefore, $d(x) = Σ_{j∈J}
λⱼ(x) d₁(xⱼ)$ implies $d(x).∇p(x) > 0$ and $∀(i,σ) ∈ I_ε(x), d(x).∇ᵢ^σp >
0$.  Thus, we have:
$$∀x ∈ \SNR, d(x).∇p(x) > 0 \hbox{ and }∀(i,σ) ∈
  I_ε(x), d(x).∇ᵢ^σp > 0.$$

  As $\SNR$ is compact, we can find $μ > 0$ such that
$$∀x ∈ \SNR, d(x).∇p(x) > μ \hbox{ and }∀(i,σ) ∈
  I_ε(x), d(x).∇ᵢ^σp > μ.$$


  To extend this to $ℝ^{n+1}∖\{0\}$, we just have to set $d(λx) = d(x)$ for $λ >
  0$ and as $p$ is homogeneous of degree $d$, we have:


  $$∀x ∈  ℝ^{n+1} ∖ \{0\}, d(x).∇p(x) > μ\|x\|^{d-1} \hbox{ and }∀(i,σ) ∈
  I_ε(x), d(x).∇ᵢ^σp > μ.$$
\end{proof}

\begin{prop}
  Let $p$ be an homogeneous polynomial of degree $d$ on
$ℝ^{n+1}$.
  Let $(Δᵢ)_{i∈I}$ be an $ε$-fat simplicial partition of $\PNR$ adapted to $p$.
  the function  $p$ and $\overline{p}$ have homeomorphic zero locus on
  $ℝ^{n+1}∖\{0\}$.
\end{prop}

\begin{proof}
  The function $\overline{p}$ as a gradient defined in any direction $d$ which
  may be defined as
  $$
  \nabla \overline{p}(x). d = \lim_{h → 0⁺} \frac{\overline{p}(x + hd) - \overline{p}(x)}{h}
  $$
  This is true because for $h > 0$ and small enough, $x + hd$ and $x$ belong to
  at least one common cone $Δᵢ^σ$ with $(i, σ) ∈ I(x)$.
  If they belong to more than one cone, the
  gradient $∇ᵢ^σ . d$ coincide and are the limit.
  This tells us moreover that we have
  $$\nabla \overline{p}(x). d = ∇ᵢ^σp . d \hbox{ for some }
  (i, σ) ∈ I(x).$$

  Next, we Consider the differential equation $x'(t) = d(x(t))$.
  This differential equation has
  unique solutions in the neighbourhood of $x$ because $d$ is $C^∞$.

  Now consider a closed and connnected region $Ω$ of $ℝ^{n+1} ∖ \{0\}$
  where $p(x) ≤ 0$ and $\overline{p}(x) ≥ 0$, and a maximal solution
  $x : L → Ω$ with $0$ in the interior of the interval $L$ and $x(0)$ in
  the interior of $Ω$.

  We define $\|p\| = max_{x ∈ \SNR} |p(x)|$ and  $\|\overline{p}\| = max_{x ∈
    \SNR} |\overline{p}(x)|$.

  From the definition of this region,
  no $x ∈ Ω$ satisfies the condition $Π_ε(x)$. Hence we have
  $d(x). ∇p(x) > 0$, so the solution is increasing in $p$.
  Similarly, $∀(i,σ) ∈ I_ε(x), d(x). ∇ᵢ^σ > 0$, hence the solution is also
  increasing in $\overline{p}$.

  Therefore, for $t < 0$, we have $|p(x(0))| < |p(x(t))| < \|p\| \|x(t)\|ᵈ$.
  for $t > 0$, we have  $|\overline{p}(x(0))| < |\overline{p}(x(t))| <
  \|\overline{p}\| \|x(t)\|$.
  Let use take $R = min(\frac{\overline{p}(x(0))}{\|\overline{p}\|},
  \left(\frac{p(x(0))}{\|p\|}\right)^{\frac{1}{d}})$,
  we have $\|x(t)\| > R$.

  Next, consider $f(t) = p(x(t))$, $f'(t) = ∇p(x(t)).d(x(t)) > μ R^{d-1} $
  Hence $-p(x(0)) > ∫_J⁺ f'(t) dt > t μ R^{d-1}$. Therefore the interval $J$ has
  an upper bound $b > 0$.

  Similarely, we considering $g(t) = \overline{p}(x(t))$, we find that
  $J$ as a lower bound $a < 0$.




  %% Thus, our maximal solution may only
  %% start at $0$, infinity or a border of $Ω$ where $\overline{p} = 0$ and may only end
  %% at $0$, infinity or a border of $Ω$ where $p = 0$.

  %% First, we rule out zero, as $p(x(t))$ is increasing and $p(x(t)) ≤ 0$ on the
  %% whole region, the solution can not start at $x = 0$, because $p(0) = 0$.  as
  %% $\overline{p}(x(t)) ≥ 0$ and $\overline{p}(x(t))$ is increasing the solution can
  %% not end at $0$.

  %% If a the solution starts at infinity, $\overline{p}(x(t))$ is increasing
  %% and $\overline{p}(x(t)) ≥ 0$, Hence when $x(t)$ converges to the beginning of
  %% the interval $I$, we have $\overline{p}(x(t))$ which converges to a finite
  %% limite. And if $x(t)$ approaches infinity ...

  %% $p(x(t))$ must have a finite limit at the begginning
  %% of the solution which implies that $∇p(x(t)).x'(t)$ should converge toward $0$
  %% when $t$ converges to the beginning of $I$. This is not possible

  %% over $ℝ^{n+1} \setminus \{0\}$, it has maximal solution. We consider maximal solution
  %% $x : [0, a[ \to ℝ^{n+1} \setminus \{0\}$ starting with $p(x(0)) = 0$, with
  %%     $x_0 \in ℝ^{n+1} \setminus \{0\}$.

  %%     We have $p(x(t))' = \nabla p(x(t)) . d(x(t)) > 0$, which means that
  %%     $p(x(t))$ is increasing.

  %%     We have $\overline{p(x(t))} = \nabla \overline{p}(x(t)). d(x(t)) > 0$
  %%     hence $\overline{p(x(t))}$ is decreasing. If $x(t)$ would approach
  %%     infinity, the it would do so with a derivate approaching the plane
\end{proof}

 \end{document}
