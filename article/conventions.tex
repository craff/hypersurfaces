\section{Conventions and notations}

We consider the real projective plane $\PNR$.
 We define $πₙ:ℝ^{n+1}∖\{0\} → \PNR$ the
projection $x ↦  \{ λx, λ∈ℝ^⋆ \} ∈ \PNR$. To simplify, we often write $\overline{x}$ for $πₙ(x)$.


We use on $\PNR$ the measure and topology induced by
the chord metric of the unit sphere:
$$d(\overline{x},\overline{y}) = \min\left(
\left\|{\frac{x}{\|x\|}} - {\frac{y}{\|y\|}}\right\|, \left\|{\frac{x}{\|x\|}} + {\frac{y}{\|y\|}}\right\|\right) (∀\overline{x},\overline{y} ∈ \PNR).$$

If $x ∈ \PNR$ (resp. $ℝ^{n+1}$), $\ball{x}{r}$ denotes the open ball of center $x$
and radius $r$ and $\cball{x}{r}$ denotes its closure.

We write $\cardinal{S}$ the cardinal of a set $S$.

For $A ⊂ ℝ^{n+1}$, $\hull{A}$ denotes the convex hull of $A$ and $\cone{A}$ its
convex cone:
$$\cone{A} = \{ ∑_{i=0}ⁿ λᵢ xᵢ, x₀,\dots,xₙ ∈ A, λ₀,\dots,λₙ∈ℝ^⋆₊\}$$
$$\hull{A} = \{ ∑_{i=0}ⁿ λᵢ xᵢ, x₀,\dots,xₙ ∈ A, λ₀,\dots,λₙ∈ℝ^⋆₊,
∑_{i=1}ⁿ λᵢ=1 \}$$

We use the following notation for simplices:
\begin{itemize}
\item If $Δ ⊂ \PNR$ is a simplex, we denote $\vertices{Δ} ⊂ \SNR$ a finite set of
  representatives of the vertices of $Δ$.
\item $\vertices{Δ}$ must verify that $Δ = πₙ(\hull{\vertices{Δ}})$. This is why we take $\vertices{Δ} ⊂
    \SNR$ instead of $\vertices{Δ} ⊂ \PNR$ as the convex hull is not well defined in
    the projective plane.
\item Simplices should not be too big, in particular, $\PNR$ is not a simplex!
  For this we require $\vertices{Δ}$ to fit in an open half space of
  $\SNR$. If this is not possible, we consider that $Δ$ is not a simplex.
\item We take $\vertices{Δ}$ to be minimal and consider only non degenerated simplex: removing one point in
    $\vertices{Δ}$
    changes it convex hull and the dimension of the
    simplex $Δ$ is always exactly $\cardinal{\vertices{Δ} - 1}$. This also means that
    $\vertices{Δ}$ is the set of extremal points of its convex hull.
  \item Because we require simplices to be not too big $πₙ^{-1}(Δ)$ has exactly two connected components:
    $$πₙ^{-1}(Δ) = \cone{\vertices{Δ}} ∪ -\cone{\vertices{Δ}}.$$
    We write $Δ⁺ =  \cone{\vertices{Δ}}$ and $Δ⁻ = - Δ⁺$. Here $-A$ denotes
    $\{-x, x ∈ A\}$.

\end{itemize}

%<!-- Local IspellDict: british -->
%<!-- Local IspellPersDict: ~/.ispell-british -->
