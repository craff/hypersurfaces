\usepackage{hyperref}
\usepackage{enumerate}
\usepackage{enumitem}
\usepackage{amssymb,amsmath,amsthm} %must be before unicode-math
\usepackage[mathletters]{ucs}
\usepackage{tikz}
\DeclareUnicodeCharacter{8348}{_t}
\DeclareUnicodeCharacter{8342}{_k}
\DeclareUnicodeCharacter{8339}{_x}
\DeclareUnicodeCharacter{8345}{_n}
\DeclareUnicodeCharacter{7522}{_i}
\DeclareUnicodeCharacter{11388}{_j}
\DeclareUnicodeCharacter{8344}{_m}
\DeclareUnicodeCharacter{7523}{_r}
\DeclareUnicodeCharacter{7580}{^c}
\DeclareUnicodeCharacter{7496}{^d}
\DeclareUnicodeCharacter{7511}{^t}
\DeclareUnicodeCharacter{7503}{^k}
\DeclareUnicodeCharacter{7488}{^T}
\DeclareUnicodeCharacter{8346}{_p}
\DeclareUnicodeCharacter{7493}{^\alpha}
\usepackage[utf8x]{inputenc}
\usepackage{graphicx}
\usepackage{stmaryrd}
\usepackage[autosize]{dot2texi}
\usepackage{multicol}
\setlength{\multicolsep}{0pt}
\usepackage{tikz}
\usetikzlibrary{shapes,arrows,tikzmark,decorations.pathreplacing,calc}
\usepackage{adjustbox}

% Title portion. Note the short title for running heads

\title{Piecewise affine approximation of smooth algebraic subvariety of the
projective space in any dimension and codimension}
\author{Christophe Raffalli}

\newcommand{\interior}[1]{%
  #1^{\mathrm{o}}%
}
\newcommand{\cardinal}[1]{%
  \#({#1})%
}
\newcommand{\hull}[1]{%
  \mathcal H({#1})%
}
\newcommand{\nhull}[2]{%
  \mathcal H^\nabla({#1},{#2})%
}
\newcommand{\cone}[1]{%
  \mathcal C({#1})%
}
\newcommand{\ncone}[2]{%
  \mathcal C^\nabla({#1},{#2})%
}
\newcommand{\vertices}[1]{%
  \mathcal V({#1})%
}
\newcommand{\ball}[2]{%
  \mathcal B_{#2}({#1})%
}
\newcommand{\cball}[2]{%
  \mathcal Bᶜ_{#2}({#1})%
}

\newcommand{\PNR}{{\cal P}^n(ℝ)}
\newcommand{\SNR}{{\cal S}^n(ℝ)}
\newcommand{\sgn}{\mathrm{sgn}}

\newtheorem{theo}{Theorem}
\newtheorem{coro}[theo]{Corollary}
\newtheorem{nota}[theo]{Notation}
\newtheorem{defi}[theo]{Definition}
\newtheorem{prop}[theo]{Proposition}
\newtheorem{exam}[theo]{Example}
\newtheorem{lemm}[theo]{Lemma}
