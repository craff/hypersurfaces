\section{A variation on Cauchy-Lipschitz theorem}

To prove our main theorem we need a theorem of existence and uniqueness
of solutions for ordinary differential equations that accepts non continuous
function for $t=0$. Here is our lemma:

\begin{lemm}
  Let $Ω$ be an open set of $\SNR$ for the topology of the sphere.
  Let $f$ be a
  function in $]0,1] × Ω → ℝ^{n+1}$ such that $f(t,x)$ is tangent to $\SNR$ at $x$. We assume that $f$ is $(L,α)$-Lipschitz for
  some $L > 0$ and $α ≥ 0$ which we define by:

  \begin{enumerate}
  \item $f$ is continuous and bounded on $]0,1] × Ω$,
  \item $∀t ∈ ]0,1], ∀x,y ∈ Ω, \|f(t,x) - f(t,y)\| ≤ \frac{L}{tᵅ} \|x - y\|$ and
  \end{enumerate}

  Then, the differential equation
  \begin{eqnarray}
    x'(t) = f(t,x(t)) \label{lipschitz-diffeq}
  \end{eqnarray}
  has unique maximal for any initial condition $x(t₀) = x₀ ∈ Ω$
  with $t₀ ∈ [0,1]$. This solutions are of class ${\cal C}¹$.

  Moreover, if we can prove that a solution can't converge to a point outside
  $Ψ$, then this solution is total, i.e. defined on $[0,1]$.
\end{lemm}

\begin{proof}

We remark that if $f$ is $(L,α)$-Lipschitz on $[0,1] × Ω$ then it
is $\frac{L}{tᵅ}$-Lipschitz on $[t,1] × Ω$ for any $t ∈ ]0,1[$.

This proves by the standard Cauchy-Lipschitz result that for any initial condition
$x(t₀) = x₀ ∈ Ω$ with $t₀ ∈ ]0,1]$, we have a unique solution on $[t,1]$ for
    any $0 < t < t₀$ and therefore a unique maximal solution.

    It remains to show that the solution is of class ${\cal C}¹$ even in $0$.
    If a maximal solution is defined on $]0,b]$, then by compacity and because
        $f(t,x)$ is bounded, both $x(t)$ and $x'(t)$ have limits for $t=0$ (this
        limit may be outside of $Ψ$).  The limit of the latter is the derivative
        of $x$ in $0$ by the usual result on the limit of derivatives.
        We do not have that $x'(0) = f(0,x(0))$ as $f$ is not even defined in $0$.

It remains to show the existance and unicity of solutions for initial conditions
in $0$.

We define $n₁(t) = t^{2L}$ and $n_α(t) = e^{-\frac{2L}{α-1} t^{1-α}}$ when $α >
1$. In all cases, we have $n(t) → 0$ when $t → 0$ and $n'_α(t) = \frac{2L}{tᵅ}
n_α(t)$. Indeed:
\begin{itemize}
\item For $α = 1$, $n'₁(t) = 2L t^{2L-1} = \frac{2L}{t} n₁(t)$
\item For $α ≥ 0, α≠ 1$, $n'_α(t) = 2L t^{-α} e^{- \frac{2L}{α-1} t^{1-α}} = \frac{2L}{tᵅ} n_α(t)$
\end{itemize}

Then, we define the
following norms on the set of continuous, partial functions from $[0,1]$ to $Ω$:
$$\|x\|_α = \max_{t ∈ ]
0,1], x(t) \rm defined} \frac{\|x(t)\|}{n_α(t)}$$

Remark: if $\|x\| = 0$ and $x(0) ≠ 0$ then, as $x$ is continuous, we can find $t
> 0$ such that $x(t) > \frac{x(0)}{2}$, hence $\|x\|_α >
\frac{x(0)}{2 n_α(t)}$ which is a contradiction. Therefore, $\|.\|_α$ is indeed
a norm.


We assume that the initial condition of the differential equation
\ref{lipschitz-diffeq} is
$x(0) = x₀ ∈ Ω$.
We define the following operator on continuous partial function from $[0,1]$ to
$Ω$:
$$F(x)(t) = x₀ + ∫^t_0 f(u,x(u)).{\rm d} u$$

For any $t ∈ ]0,1]$, we have:
\begin{eqnarray*}
  \|F(x)(t) - F(y)(t)\| &≤& ∫^t_0 \|f(u,x(u)) - f(u,y(u))\| {\rm d} u \cr
  &≤& ∫^t_0 \frac{L}{uᵅ} \|x(u) - y(u)\| {\rm d} u \cr
  &≤& \|x - y\|_α ∫^t_0 \frac{L}{uᵅ} n_α(u)  {\rm d} u \cr
  &≤& \|x - y\|_α ∫^t_0 \frac{1}{2} n'_α(u)  {\rm d} u \cr
  &≤&\frac{1}{2} \|x - y\|_α n_α(t) \cr
\|F(x) - F(y)\|_α &≤& \frac{1}{2} \|x - y\|_α
\end{eqnarray*}

Therefore, $F$ is
contracting, this ensures the existence and unicity of a fixpoint of $F$, hence
of a solution of the differentiel equation \ref{lipschitz-diffeq} with initial
condition $x(0) = x₀$. Again by compacity and boundedness of $f$ this solution
is of class ${\cal C}¹$.

\end{proof}

%% We choose $s : ℝ^{n+1} → ℝ$, a $C^∞$ function with support in
%% $\ball{0}{1}$ and $∫_{ℝ^{n+1}} s(t) dt = 1$ for the standard measure $dt$
%% on $ℝ^{n+1}$. We will use the same function $s$ in the rest of this paper.

%% \begin{lemm}
%% Let $g$ be a homogeneous continuous function of degree $d$ in $ℝ^{n+1} → ℝ$. We define
%% $f(t,x) = ∫_{ℝ^{n+1}} g(x - t ε u) s(u) {\rm d} u$.
%% Then, $\nablaₓ f$ is L-Lipschitz¹ on $\SNR$.
%% \end{lemm}

%% \begin{proof}
%%   We have
%%   \begin{eqnarray*}
%%     f(t,x) &=& ∫_{ℝ^{n+1}} g(x - t ε u) s(u) {\rm d} u \cr
%%     &=& \frac{1}{(tε)^{n+1}} ∫_{ℝ^{n+1}} g(v) s(\frac{x - v}{tε}) {\rm d} v \cr
%%     \nablaₓ f(t,x) &=&
%%     \frac{1}{(tε)^{n+2}} ∫_{ℝ^{n+1}} g(v) ∇s(\frac{x - v}{tε}) {\rm d} v \cr
%%   \end{eqnarray*}
%% \end{proof}


%<!-- Local IspellDict: british -->
%<!-- Local IspellPersDict: ~/.ispell-british -->
