\section{Simplicial partition}

\begin{defi}
  A \emph{simplicial partition} of
  $\PNR$ is a finite set of closed simplices $(Δᵢ)_{i∈I}$ such that
  \begin{itemize}
  \item Each $Δᵢ$ is a simplex of dimension $n$, i.e. $\cardinal{\vertices{Δ}} = n + 1$.
  \item $⋃_{i∈I} Δᵢ = \PNR$,
  \item For all $i₁ < i₂ < \dots iₖ ∈ I$, $Δ_{i₁} ∩ \dots ∩ Δ_{iₖ}$ is empty or is a simplex of dimension at most
    $n-k+1$ and $πₙ(\vertices{Δ_{i₁} ∩ \dots ∩ Δ_{iₖ}}) =
    πₙ(\vertices{Δ_{i₁}}) ∩ \dots ∩ πₙ(\vertices{Δ_{iₖ}})$.
    We can not require  $\vertices{Δᵢ ∩ Δⱼ} = \vertices{Δᵢ} ∩ \vertices{Δⱼ}$ because we may have $x ∈
    \vertices{Δᵢ}$ and $y ∈ \vertices{Δⱼ}$ with $x ≠ y$ and $\overline{x} = \overline{y}$ (see
    the example below). This conditions ensures that an intersection of
    simplices consists exactly in a face of lower dimensions defined by the
    common vertices of the simplices.
%  \item We say that $Δᵢ$ and $Δⱼ$ are neighbour if $Δᵢ ∩ Δⱼ ≠ ∅$. The neighbourhood relation is clearly reflexive and symmetric.
  \end{itemize}
\end{defi}

Here is as example a partition of $\PNR$ with $2ⁿ$ simplices:
\begin{exam}\label{init_part}
Consider $B = \{x₀,\dots,xₙ\}$, the canonical base of $ℝ^{n+1}$ and
$(εᵢ)_{i ∈ \{1,\dots,2ⁿ\}}$ an enumeration of all sequences of length $n$ of
$1$ or $-1$. Then, we define $(Δᵢ)_{i∈\{1,\dots,2ⁿ\}}$ by
$$Δᵢ = πₙ(\hull{\{x₀,ε_{i,1} x₁, \dots, ε_{i,n} xₙ\}}).$$
\end{exam}

We remark in this example, that all simplices use $B$ as set of vertices. This
means that for all $i < j ∈ \{1,\dots,2ⁿ\}$, we have
  $πₙ(\vertices{Δᵢ}) = πₙ(\vertices{Δⱼ}) = πₙ(B)$ while $\vertices{Δᵢ} ≠ \vertices{Δⱼ}$.

In this examples, we have
\begin{eqnarray*}
  Δ⁺ᵢ &=& \cone{\{x₀,ε_{i,1} x₁, \dots, ε_{i,n} xₙ\}} \cr
  Δ⁻ᵢ &=& \cone{\{-x₀,-ε_{i,1} x₁, \dots, -ε_{i,n} xₙ\}} \cr
\end{eqnarray*}

\begin{defi} Let $(Δᵢ)_{i∈I}$ be a simplicial partition of $\PNR$. For $x ∈
  ℝ^{n+1} ∖\{0\}$ we define
  $I^Δ(x) = \{(i,σ) ∈ I × \{-,+\}, x ∈ Δᵢ^σ\}$.
  We also define $I^Δ(x,ε) = \{(i,σ) ∈ I × \{-,+\}, \ball(x,ε) ∩ Δᵢ^σ ≠ \{\}\}$.

  In general we only use one
  simplicial partition at a time and we simply write $I(x)$ and $I(x,ε)$.
\end{defi}

\begin{prop}\label{Ihomo}
  Let $(Δᵢ)_{i∈I}$ be a simplicial partition of $\PNR$.
  For $x ∈  ℝ^{n+1} ∖\{0\}$ and $λ > 0$, we have $I(λx) = I(x)$ and $I(λx,λε) = I(x,ε)$.
\end{prop}

\begin{proof}
  From the fact that the set $Δᵢ^σ$ are convex cones.
\end{proof}

\begin{defi}
  Let $(Δᵢ)_{i∈I}$ be a simplicial partition of $\PNR$.
  We say that it is $ε$-fat if for any $x ∈ ℝ^{n+1} ∖\{0\}$
  we have
   $$ ⋂_{(i,σ) ∈ I(x,ε)} Δᵢ^σ ≠ ∅$$
\end{defi}

\begin{lemm}
  Let $(Δᵢ)_{i∈I}$ be an $ε$-fat  simplicial partition of $\PNR$.
  For any $x ∈ ℝ^{n+1} ∖\{0\}$,
  there exists $y ∈
  ℝ^{n+1} ∖\{0\}$, such that $I(x,ε\|x\|) = I(y)$.

\end{lemm}

\begin{proof}
  First, we have $I(x,ε\|x\|) = I(x',ε)$ with $x' = \frac{x}{\|x\|}$.
  Second, we know by definition of $ε$-fatness that $ F = ⋂_{(i,σ) ∈ I(x,ε)} Δᵢ^σ ≠
  ∅$. By definition of simplicial partition, $πₙ(F)$ if
  the common face to all simplicies $Δᵢ$. It suffices to take $y ∈ F$ such that
  $πₙ(y)$ not in a face of lower dimension than $πₙ(F)$.
\end{proof}

\begin{prop} If a simplicial partition $(Δᵢ)_{i∈I}$ is $ε$-fat for $ε > 0$
  then it is $η$-fat if $0 < η ≤ ε$.
\end{prop}

\begin{proof} From the fact that for $x ∈ ℝ^{n+1} ∖\{0\}$, $I(x,η) ⊆ I(x,ε)$ if $η ≤ ε$.
\end{proof}

\begin{prop}
  Any \emph{simplicial partition} $(Δᵢ)_{i∈I}$ of $\PNR$ is $ε$-fat for
  some $ε > 0$ small enough.
\end{prop}

\begin{proof}
  First, thanks to proposition \ref{Ihomo}, we only have to check the condition
  for $ε$-fatness in $\SNR$.

  Consider a set $J ⊂ I × \{-,+\}$ with $⋂_{(i,σ)∈J} Δⱼ^σ = ∅$ (1), we first
  prove that there exists $ε_J > 0$ such that for all $x ∈ \SNR$, $J \not⊆
  I(x,ε_J)$. For this we assume the contrary: For any $ε>0$ there exists $x_ε
  ∈ \SNR$ such that $J ⊆ I(x_ε,ε)$. As $\SNR$ is compact, we can extract a
  sequence $(εₙ,xₙ)_{n ∈ \mathbb N}$ which converges to $(0,x)$ for some $x ∈
  \SNR$, with $J ⊆ I(x_n,εₙ)$ for all $n ∈ \mathbb N$. Thus, for any $(i,σ)
  ∈ J$ and $n ∈ \mathbb N$, we have $\ball{xₙ}{εₙ} ∩ Δᵢ^σ ≠ ∅$ which implies
  $d(xₙ,Δᵢ^σ) < εₙ$. As $Δᵢ^σ$ are closed convex cone, this implies that $x ∈
  Δᵢ^σ$ for any $(i,σ) ∈ J$ which contradicts the hypothesis (1).

  We end the proof by remarking that the simplicial partition is $ε$-fat if we
  take $ε$ the minimum of all $ε_J$ for $J ⊂ I$ with $∩_{(i,σ)∈J} Δᵢ^σ =
  ∅$. This is well defined as $I$ hence his power-set are finite. Indeed, if $J
  = I(x,ε)$, and $∩_{(i,σ)∈J} Δᵢ^σ = ∅$, then $J \not ⊆ I(x,ε_J) ⊇ I(x,ε)
  = J$ which is contradictory.
\end{proof}


%<!-- Local IspellDict: british -->
%<!-- Local IspellPersDict: ~/.ispell-british -->
