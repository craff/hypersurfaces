\section{Simplicial partition}

\begin{defi}
  A \emph{simplicial partition} of
  $\PNR$ is a finite set of closed simplices $(Δᵢ)_{i∈I}$ such that
  \begin{itemize}
  \item Each $Δᵢ$ is a simplex of dimension $n$, i.e. $\cardinal{\vertices{Δ}} = n + 1$.
  \item $⋃_{i∈I} Δᵢ = \PNR$,
  \item For all $i₁ < i₂ < \dots iₖ ∈ I$, $Δ_{i₁} ∩ \dots ∩ Δ_{iₖ}$ is empty or is a simplex of dimension at most
    $n-k+1$ and $πₙ(\vertices{Δ_{i₁} ∩ \dots ∩ Δ_{iₖ}}) =
    πₙ(\vertices{Δ_{i₁}}) ∩ \dots ∩ πₙ(\vertices{Δ_{iₖ}})$.
    We can not require  $\vertices{Δᵢ ∩ Δⱼ} = \vertices{Δᵢ} ∩ \vertices{Δⱼ}$ because we may have $x ∈
    \vertices{Δᵢ}$ and $y ∈ \vertices{Δⱼ}$ with $x ≠ y$ and $\overline{x} = \overline{y}$ (see
    the example below). This conditions ensures that the intersection of
    simplices are exactly faces of lower dimensions defined by the common vertices.
%  \item We say that $Δᵢ$ and $Δⱼ$ are neighbour if $Δᵢ ∩ Δⱼ ≠ ∅$. The neighbourhood relation is clearly reflexive and symmetric.
  \end{itemize}
\end{defi}

Here is as example a partition of $\PNR$ with $2ⁿ$ simplices:
\begin{exam}\label{init_part}
Consider $B = \{x₀,\dots,xₙ\}$, the canonical base of $ℝ^{n+1}$ and
$(εᵢ)_{i ∈ \{1,\dots,2ⁿ\}}$ an enumeration of all sequences of length $n$ of
$1$ or $-1$. Then, we define $(Δᵢ)_{i∈\{1,\dots,2ⁿ\}}$ by
$$Δᵢ = πₙ(\hull{\{x₀,ε_{i,1} x₁, \dots, ε_{i,n} xₙ\}}).$$
\end{exam}

We remark in this example, that all simplices use $B$ as set of vertices. This
means that for all $i < j ∈ \{1,\dots,2ⁿ\}$, we have
  $πₙ(\vertices{Δᵢ}) = πₙ(\vertices{Δⱼ}) = πₙ(B)$ while $\vertices{Δᵢ} ≠ \vertices{Δⱼ}$.

In this examples, we have
\begin{eqnarray*}
  Δ⁺ᵢ &=& \cone{\{x₀,ε_{i,1} x₁, \dots, ε_{i,n} xₙ\}} \cr
  Δ⁻ᵢ &=& \cone{\{-x₀,-ε_{i,1} x₁, \dots, -ε_{i,n} xₙ\}} \cr
\end{eqnarray*}

\begin{defi} Let $(Δᵢ)_{i∈I}$ be a simplicial partition of $\PNR$. For $x ∈
  ℝ^{n+1} ∖\{0\}$ we define
  $I^Δ(x) = \{(i,σ) ∈ I × \{-,+\}, x ∈ Δᵢ^σ\}$. In general we only use one
  simplicial partition at a time and we simply write $I(x)$.
\end{defi}

\begin{prop}\label{Ihomo}
  Let $(Δᵢ)_{i∈I}$ be a simplicial partition of $\PNR$.
  For $x ∈  ℝ^{n+1} ∖\{0\}$ and $λ > 0$, we have $I(λx) = I(x)$.
\end{prop}

\begin{proof}
  From the fact that the set $Δᵢ^σ$ are convex cones.
\end{proof}

%<!-- Local IspellDict: british -->
%<!-- Local IspellPersDict: ~/.ispell-british -->
