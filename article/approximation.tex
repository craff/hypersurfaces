\section{Δ-linear approximation of a polynomial}

\begin{defi}
Let $p$ be an homogeneous polynomial of degree $d$ on
$ℝ^{n+1}$. Let $(Δᵢ)_{i∈I}$ be a simplicial partition of $\PNR$.
We define $\overline{p} : ℝ^{n+1} → ℝ$, the \emph{$Δ$-linear approximation of $p$},
the piecewice linear function
that is linear on all $Δ^σᵢ$ for $i ∈ I$, $σ ∈ \{+,-\}$, and equal to $p$
on all points in $\vertices{Δ}$.

For $i ∈ I$, $σ ∈ \{+,-\}$, we define $∇ᵢ^σp = ∇\overline{p}(x)$
for any $x ∈ Δ^σᵢ$, as the gradient is constant over $Δ^σᵢ$ for a linear function.
\end{defi}

Remark: if $d$ is even (resp. odd), $\overline{p}$ is even (resp. odd).

\begin{defi}
Let $p$ be an homogeneous polynomial of degree $d$ on
$ℝ^{n+1}$. Let $(Δᵢ)_{i∈I}$ be a simplicial partition of $\PNR$.
Let $\overline{p}$ be the $Δ$-linear approximation of $p$.
We define the condition $Π(x)$ to be true if and only if $p(x)$ and
$\overline{p}(x)$ are both positive or both negative.
\end{defi}

\begin{prop}\label{adaptedeq}
Let $p$ be an homogeneous polynomial of degree $d$ on
$ℝ^{n+1}$. Let $(Δᵢ)_{i∈I}$ be a simplicial partition of $\PNR$.
Let $\overline{p}$ be the $Δ$-linear approximation of $p$.

The following conditions are equivalent:
\begin{enumerate}
\item $∀x ∈  ℝ^{n+1}∖\{0\}, Π(x) \hbox{ or } 0 ∉ \cone{\{∇p(x)\} ∪ \{∇ᵢ^σp, (i,σ) ∈ I(x)\}}$
\item $∀x ∈  ℝ^{n+1}∖\{0\}, Π(x) \hbox{ or } 0 ∉ \hull{\{∇p(x)\} ∪ \{∇ᵢ^σp, (i,σ) ∈ I(x)\}}$
\item $∀x ∈  \SNR, Π(x) \hbox{ or } 0 ∉ \cone{\{∇p(x)\} ∪ \{∇ᵢ^σp, (i,σ) ∈ I(x)\}}$
\item $∀x ∈  \SNR, Π(x) \hbox{ or } 0 ∉ \hull{\{∇p(x)\} ∪ \{∇ᵢ^σp, (i,σ) ∈ I(x)\}}$
\item\label{hb} $∀x ∈  ℝ^{n+1}∖\{0\}, Π(x) \hbox{ or } ∃δ ∈ ℝ^{n+1}∖\{0\},
  \left\{\begin{array}{l} δ.∇p(x) > 0 \hbox{ and}\cr
    ∀(i,σ) ∈ I(x), δ.∇ᵢ^σp > 0
  \end{array}\right.$
\end{enumerate}
\end{prop}

\begin{proof}
  The equivalences (1) ↔ (2) and (3) ↔ (4) are a general property of convex hull
  and convex cone: for any set $X ⊂ ℝ^{n+1}$, $0 ∈ \hull{X} ↔ 0 ∈ \cone{X}$.
  %Left to right is because $\hull{X} ⊆ \cone{X}$. For right to left, if $0 ∈
  %\cone{X}$, if means we write $0 = λᵢ x₁ + \dots + λₙ xₙ$ with $λᵢ > 0$ and
  %$xᵢ ∈ X$ for $1 < i < n$. Hence, taking $μᵢ = \frac{λᵢ}{λᵢ + \dots + λₙ}$, we
  %have $0 = μᵢ x₁ + \dots + μₙ xₙ$ with $μ₁ + \dots + μₙ = 1$ and $μᵢ > 0$, $xᵢ
  %∈ X$ for $1 < i < n$.


  The equivalence (1)  ↔ (3) comes from proposition $\ref{Ihomo}$ and the fact that $p$ is homogeneous hence
  $\nabla(p)(λx) = λ^{d-1}\nabla(p)(x)$ for $λ > 0$.

  The equivalence (2) ↔ (5) is a form of Han-Banach theorem ensuring the
  existence of a supporting hyper-plane for points outside a convex body. The
  vector $δ$ is a normal vector of this supporting hyperplane \cite{Zal02}.
\end{proof}

\begin{defi}
Let $p$ be an homogeneous polynomial of degree $d$ on
$ℝ^{n+1}$. Let $(Δᵢ)_{i∈I}$ be a simplicial partition of $\PNR$.

We say that the partition is \emph{adapted} to $p$ if it satisfies the condition
of proposition \ref{adaptedeq}.
\end{defi}

\begin{prop}
  Let $p$ be an homogeneous polynomial of degree $d$ on
$ℝ^{n+1}$.
  Let $(Δᵢ)_{i∈I}$ be a simplicial partition of $\PNR$ adapted to $p$.
  We can find a $C^∞$ function $δ$ in $ℝ^{n+1} ∖ \{0\} → ℝ^{n+1}∖\{0\} $
  and $μ > 0$ such that:
  $$∀x ∈  ℝ^{n+1}∖\{0\}, Π(x) \hbox{ or } \left\{
  \begin{array}{l}δ(x).∇p(x) > μ\|x\|^{d-1} \hbox{ and } \cr ∀(i,σ) ∈
  I(x), δ(x).∇ᵢ^σp > μ.\end{array}\right.
  $$
\end{prop}

\begin{proof}
From proposition \ref{adaptedeq}.\ref{hb}, for any $x ∈ ℝ^{n+1}∖\{0\}$, if
$Π_ε(x)$ is not satisfied, we find $δ₀(x)
∈ ℝ^{n+1}$ such that $δ₀(x).∇p(x) > 0$ and $∀(i,σ) ∈ I_ε(x), δ₀(x).∇ᵢ^σp > 0$.
We can also assume that $δ₀(λ x) = δ₀(x)$ for $λ > 0$ for the same reason as in
the proof of $(1) ↔ (3)$ above.

We start with the following lemma: for each $x ∈
ℝ^{n+1}∖\{0\}$, we find $0 < η(x) < ε$ such that
$∀y ∈ \ball{x}{2η(x)}$, $d₀(x).∇p(y) > 0$ and $∀(i,σ) ∈ I_ε(y),
d₀(x).∇ᵢ^σp > 0$.

First, $(i,σ) ∈ I_ε(y) ↔ \cball{y}{ε} ∩ Δᵢ^σ ≠ ∅$ is a closed condition and the
contrary is an open condition. The condition $Π_ε(x)$ is also an open
condition. Hence, for $x ∈ℝ^{n+1}∖\{0\}$, we find $η > 0$
small enough to have $y ∈
\ball{x}{2η}$ gives $Π_ε(x)$ or $(i,σ) ∉ I_ε(x)$ implies $(i,σ) ∉ I_ε(y)$.

which means that
for $y ∈ \ball{x}{2η}$, we have $Π_ε(x)$ or $I_ε(y) ⊆ I_ε(x)$ and therefore,

$$∀y ∈ \ball{x}{2η(x)}, Π_ε(x) \hbox{ or } ∀(i,σ) ∈ I_ε(y), δ₀(x).∇ᵢ^σp > 0.$$

Second, $∇p(x)$ is continuous, thus for  $x ∈ℝ^{n+1}∖\{0\}$, we find $η > 0$
small enough to have $y ∈ \ball{x}{2η}$
implies $Π_ε(x)$ or $δ₀(x).∇p(y) > 0$. Grouping both results yields the wanted real $η(x)$
such that

$$∀x ∈ℝ^{n+1}∖\{0\}, ∀y ∈ \ball{x}{2η(x)}, Π_ε(x) \hbox{ or }
\left\{\begin{array}{l} δ₀(x).∇p(y) > 0 \hbox{ and}\cr
∀(i,σ) ∈ I_ε(y), δ₀(x).∇ᵢ^σp > 0
\end{array}\right.
$$


Now, for $x ∈ \SNR$, the open balls $\ball{x}{η(x)}$ cover $\SNR$, so we can
find a finite family $(xⱼ)_{j∈J}$ such that the balls $\ball{x}{η(xⱼ)}$ cover
$\SNR$. We also take $η = \min_{j ∈ J} η(xⱼ)$.

Next, for each $xⱼ ∈ \SNR$, $j ∈ J$, we choose a measurable set
$Dⱼ ⊂ \ball{x}{η(xⱼ)}$ such that $\SNR$ is a disjoin union of all
$Dⱼ$. And we define $d₁ : \SNR → ℝ^{n+1}∖\{0\}$ with $δ₁(x) = δ₀(xⱼ)$ if $x ∈ Dⱼ$.

We choose $s : ℝ^{n+1} → ℝ$, a $C^∞$ function with support in
$\ball{0}{1}$ and $∫_{ℝ^{n+1}} s(t) dt = 1$ for the standard measure $dt$
on $ℝ^{n+1}$.

We define

$$ δ : \SNR → ℝ^{n+1}∖\{0\} \hbox{ with } δ(x) = ∫_{ℝ^{n+1}} δ₁(x - η t)
s(t) dt$$

Similarly to a convolution product, using the change of variable $u = x - ηt$, we have $δ(x) = ∫_{ℝ^{n+1}} η^{n +1} δ₁(u)
s(\frac{x - u}{η}) du$ and therefore $δ$ is $C^∞$ on $\SNR$.

Moreover, if we take $x ∈ \SNR$, we can write
$$ δ(x) = Σ_{j∈J} δ₁(xⱼ) ∫_{x - η t ∈ Dⱼ} s(t) dt$$

Let us define $λⱼ(x) = ∫_{x - η t ∈ Dⱼ} s(t) dt$. As $Dⱼ ⊂ \ball{xⱼ}{η(xⱼ)}$,
if $λⱼ(x) > 0$, we have $d(x - η t,xⱼ) ≤ η(xⱼ)$ otherwise $x  - η t ∉ Dⱼ$
and $d(x, x - η t) < η ≤ η(xⱼ)$ otherwise $s(t) = 0$. Therefore,
if $λⱼ(x) > 0$, we have $d(x,xⱼ) < 2η(xⱼ)$.
which implies $Π_ε(x)$ or both $δ₁(xⱼ).∇p(x) > 0$ and $∀(i,σ) ∈ I_ε(x), δ₁(xⱼ).∇ᵢ^σp > 0$.

Moreover, the coefficient $λⱼ$ are positive. Therefore, $δ(x) = Σ_{j∈J}
λⱼ(x) δ₁(xⱼ)$ implies $δ(x).∇p(x) > 0$ and $∀(i,σ) ∈ I_ε(x), δ(x).∇ᵢ^σp >
0$.  Thus, we have:
$$∀x ∈ \SNR, Π_ε(x) \hbox{ or } δ(x).∇p(x) > 0 \hbox{ and }∀(i,σ) ∈
  I_ε(x), δ(x).∇ᵢ^σp > 0.$$

  As $\SNR$ is compact, we can find $μ > 0$ such that
$$∀x ∈ \SNR, Π_ε(x) \hbox{ or } δ(x).∇p(x) > μ \hbox{ and }∀(i,σ) ∈
  I_ε(x), δ(x).∇ᵢ^σp > μ.$$


  To extend this to $ℝ^{n+1}∖\{0\}$, we just have to set $δ(λx) = δ(x)$ for $λ >
  0$ and as $p$ is homogeneous of degree $d$, we have:


  $$∀x ∈  ℝ^{n+1} ∖ \{0\}, Π_ε(x) \hbox{ or } \left\{\begin{array}{l}
  δ(x).∇p(x) > μ\|x\|^{d-1} \hbox{ and }\cr∀(i,σ) ∈
  I_ε(x), δ(x).∇ᵢ^σp > μ
  \end{array}\right.$$
\end{proof}

\begin{prop}
  Let $p$ be an homogeneous polynomial of degree $d$ on
$ℝ^{n+1}$.
  Let $(Δᵢ)_{i∈I}$ be an $ε$-fat simplicial partition of $\PNR$ adapted to $p$.
  the function  $p$ and $\overline{p}$ have homeomorphic zero locus on
  $ℝ^{n+1}∖\{0\}$.
\end{prop}

\begin{proof}
  The function $\overline{p}$ as a gradient defined in any direction $d$ which
  may be defined as
  $$
  \nabla \overline{p}(x). d = \lim_{h → 0⁺} \frac{\overline{p}(x + hd) - \overline{p}(x)}{h}
  $$
  This is true because for $h > 0$ and small enough, $x + hd$ and $x$ belong to
  at least one common cone $Δᵢ^σ$ with $(i, σ) ∈ I(x)$.
  If they both belong to more than one cone, as we take the gradient of linear
  functions that coincide on the intersection of several simplex in a direction which
  is parallel to this intersection, the gradient $∇ᵢ^σp . d$ coincide and are the limit.
  This tells us moreover that we have
  $$\nabla \overline{p}(x). d = ∇ᵢ^σp . d \hbox{ for some }
  (i, σ) ∈ I(x).$$

  Next, we Consider the differential equation $x'(t) = δ(x(t))$.
  This differential equation has
  unique solutions in the neighbourhood of $x$ because $δ$ is $C^∞$.

  Now consider a closed and connnected region $Ω$ of $ℝ^{n+1} ∖ \{0\}$
  where $p(x) ≤ 0$ and $\overline{p}(x) ≥ 0$, and a maximal solution
  $x : L → Ω$ with $0$ in the interior of the interval $L$ and $x(0)$ in
  the interior of $Ω$. The same reasoning will apply in region where $p(x) ≥ 0$
  and $\overline{p}(x) ≤ 0$.

  We define $\|p\| = max_{x ∈ \SNR} |p(x)|$ and  $\|\overline{p}\| = max_{x ∈
    \SNR} |\overline{p}(x)|$.

  From the definition of this region,
  no $x ∈ Ω$ satisfies the condition $Π_ε(x)$. Hence we have
  $d(x). ∇p(x) > μ\|x\|^{d-1} > 0$, so the solution is increasing in $p$.
  Similarly, $∀(i,σ) ∈ I_ε(x), d(x). ∇ᵢ^σ > μ > 0$, hence the solution is also
  increasing in $\overline{p}$.

  Therefore, for $t < 0$, we have $|p(x(0))| < |p(x(t))| < \|p\| \|x(t)\|ᵈ$.
  for $t > 0$, we have  $|\overline{p}(x(0))| < |\overline{p}(x(t))| <
  \|\overline{p}\| \|x(t)\|$.
  Let use take $R = min(\frac{\overline{p}(x(0))}{\|\overline{p}\|},
  \left(\frac{p(x(0))}{\|p\|}\right)^{\frac{1}{d}})$,
  we have $\|x(t)\| > R$.   This exludes solution converging to $0$.

  Next, consider $f(t) = p(x(t))$, $f'(t) = ∇p(x(t)).δ(x(t)) > μ R^{d-1} $
  Therefore $f(t) - f(0) >  ∫₀ᵗ f'(t) dt > t μ R^{d-1}$. Therefore $p(x(t)$
  reaches the zero locus of $p$ for $0 < t < - f(0) / μ R^{d-1}$.
  It will also reach  the zero locus of $\overline{p}$ for $0 > t > - f(0) / μ$.
  This excludes all solution with unbounded interval definition.

  From this we can define the homeomorphism $ϕ$ between the zero loci of $p$ and
  and $\overline{p}$. Consider a point $x₀$ such that $p(x₀) = 0$. If
  $\overline{p}(x₀) = 0$, we take $ϕ(x₀) = x₀$. If $\overline{p}(x₀) < 0$, we
  consider the solution of the equation $x'(t) = δ(x(t))$, with $x(0) = x₀$.
  This solution reaches $\overline{p}(x(t))$ for some $t₁ < 0$ and we take
  $ϕ(x₀) = x(t₁)$. If $\overline{p}(x₀) > 0$, we
  consider the solution of the equation $x'(t) = δ(x(t))$, with $x(0) = x₀$.
  This solution reaches $\overline{p}(x(t))$ for some $t₁ > 0$ and we take
  $ϕ(x₀) = x(t₁)$.

  The function $ϕ$ is bijective, because its inverse can be constructed in the
  same way and solution of the differential equation is unique. And it is
  $C^∞$ because $δ$ is $C^∞$.
\end{proof}

%<!-- Local IspellDict: british -->
%<!-- Local IspellPersDict: ~/.ispell-british -->
