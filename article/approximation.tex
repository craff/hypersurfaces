\section{Δ-linear approximation of a polynomial}

\begin{defi}
Let $p$ be an homogeneous polynomial of degree $d$ on
$ℝ^{n+1}$. Let $(Δᵢ)_{i∈I}$ be a simplicial partition of $\PNR$.
We define $\overline{p} : ℝ^{n+1} → ℝ$, the \emph{$Δ$-linear approximation of $p$},
the piecewice linear function
that is linear on all $Δ^σᵢ$ for $i ∈ I$, $σ ∈ \{+,-\}$, and equal to $p$
on all points in $\vertices{Δ}$.

For $i ∈ I$, $σ ∈ \{+,-\}$, we define $∇ᵢ^σp = ∇\overline{p}(x)$
for any $x ∈ Δ^σᵢ$, as the gradient is constant over $Δ^σᵢ$ for a linear
function.

We also define $\hat p(x) = \sgn(\overline{p}(x))^{d-1} \overline{p}^d(x)$. $\hat p$
is homogeneous of degree $d$, i.e. for $x \in ℝ^{n+1}$ and $λ>0$, $\hat p(λx) = λᵈ
\hat p(x)$. The function $\overline{p}$ and $\hat p$ obviously have the same sign and
zero locus on $ℝ^{n+1}$.
\end{defi}

\begin{lemm}
Let $p$ be an homogeneous polynomial of degree $d$ on
$ℝ^{n+1}$. Let $(Δᵢ)_{i∈I}$ be a simplicial partition of $\PNR$.
The function $\overline{p}$ and $\hat p$ have derivatives at $x ∈ ℝ^{n+1} ∖\{0\}$ in the
direction $v$ if the point $I(x + h v)$ is constant
for $h ∈ [-ε,ε]$ with $ε > 0$ is small enough.

This means that for any line segment $\overline{p}$ and $\hat p$ have
derivatives almost everywhere in the direction of this line.
\end{lemm}

\begin{proof}
  Lines can only meet transversaly intersections $πₙ^{-1}$ of the intersection
  of at least two simplices of $Δ$ in finitely many points.
\end{proof}

\begin{lemm}
  From the previous lemma, with the same hypothesis, we find that $\hat p$ is
  Lipschitz in $\ball{0}{M}$ for any $M ∈ ℝ₊$.
\end{lemm}

\begin{proof}
  Let $L$ be the maximum of the differential of $\hat p$, where it is defined in
  $\ball{0}{M}$. At a point $x$ in $D = \ball{0}{M} ∩ πₙ^{-1} (Δ_{i₁} ∩ \dots ∩
  Δ_{iₖ})$, if $v$ is a direction tangent to this intersection, then $\nabla
  \hat p_{i₁}(x) . v = \dots = \nabla \hat p_{iₖ}(x) . v$ because the functions
  $\hat p_{i₁}, \dots, \hat p_{iₖ}$ coincide on $D$. We will denote this number
  by $\nabla \hat p(x) . v$, which for $v$ fixed is defined almost every where,
  and we have $\nabla \hat p(x) . v ≤ \min_{j ∈ (i₁,\dots,iₖ)} \|\nabla \hat
  pⱼ(x)\| \|v\| ≤ L \|v\|$.

  Then, we have
  \begin{eqnarray*}
    |\hat p(x) - \hat p(y)| &=& \left. ∫₀¹ \nabla \hat p(x + (1-t) y) . (x - y)
    dt\right. \cr
    &≤&   ∫₀¹ L \|(x - y)\| dt \cr
    &≤& L \|x - y\|
  \end{eqnarray*}
\end{proof}

\begin{defi}
Let $p$ be an homogeneous polynomial of degree $d$ on
$ℝ^{n+1}$. Let $(Δᵢ)_{i∈I}$ be a simplicial partition of $\PNR$.
Let $\overline{p}$ be the $Δ$-linear approximation of $p$.
We define
\begin{itemize}
  \item the condition $Π_p(x)$ to be true if and only if $p(x)$ and
$\overline{p}(x)$ are both positive or both negative (i.e. $p(x)
    \overline{p}(x) > 0$),
  \item The set $\ncone{p}{x} = \cone{\{∇p(x)\} ∪ \{∇ᵢ^σp, (i,σ) ∈ I(x)\}}$
  \item The set $\nhull{p}{x} = \hull{\{∇p(x)\} ∪ \{∇ᵢ^σp, (i,σ) ∈ I(x)\}}$
\end{itemize}
\end{defi}

\begin{prop}\label{adaptedeq}
Let $p$ be an homogeneous polynomial of degree $d$ on
$ℝ^{n+1}$. Let $(Δᵢ)_{i∈I}$ be a simplicial partition of $\PNR$.
Let $\overline{p}$ be the $Δ$-linear approximation of $p$.

The following conditions are equivalent:
\begin{enumerate}
\item $∀x ∈  ℝ^{n+1}∖\{0\}, Π_p(x) \hbox{ or } 0 ∉ \ncone{p}{x}$
\item $∀x ∈  ℝ^{n+1}∖\{0\}, Π_p(x) \hbox{ or } 0 ∉ \nhull{p}{x}$
\item $∀x ∈  \SNR, Π_p(x) \hbox{ or } 0 ∉ \ncone{p}{x}$
\item $∀x ∈  \SNR, Π_p(x) \hbox{ or } 0 ∉ \nhull{p}{x}$
\item\label{hb} $∀x ∈  ℝ^{n+1}∖\{0\}, Π_p(x) \hbox{ or } ∃δ ∈ ℝ^{n+1}∖\{0\},
  \left\{\begin{array}{l} δ.∇p(x) > 0 \hbox{ and}\cr
    ∀(i,σ) ∈ I(x), δ.∇ᵢ^σp > 0
  \end{array}\right.$
\end{enumerate}
\end{prop}

\begin{proof}
  The equivalences (1) ↔ (2) and (3) ↔ (4) are a general property of convex hull
  and convex cone: for any set $X ⊂ ℝ^{n+1}$, $0 ∈ \hull{X} ↔ 0 ∈ \cone{X}$.
  %Left to right is because $\hull{X} ⊆ \cone{X}$. For right to left, if $0 ∈
  %\cone{X}$, if means we write $0 = λᵢ x₁ + \dots + λₙ xₙ$ with $λᵢ > 0$ and
  %$xᵢ ∈ X$ for $1 < i < n$. Hence, taking $μᵢ = \frac{λᵢ}{λᵢ + \dots + λₙ}$, we
  %have $0 = μᵢ x₁ + \dots + μₙ xₙ$ with $μ₁ + \dots + μₙ = 1$ and $μᵢ > 0$, $xᵢ
  %∈ X$ for $1 < i < n$.


  The equivalence (1)  ↔ (3) comes from proposition $\ref{Ihomo}$ and the fact that $p$ is homogeneous hence
  $\nabla(p)(λx) = λ^{d-1}\nabla(p)(x)$ for $λ > 0$.

  The equivalence (2) ↔ (5) is a form of Han-Banach theorem ensuring the
  existence of a supporting hyper-plane for points outside a convex body. The
  vector $δ$ is a normal vector of this supporting hyperplane \cite{Zal02}.
\end{proof}

\begin{defi}
Let $p$ be an homogeneous polynomial of degree $d$ on
$ℝ^{n+1}$. Let $(Δᵢ)_{i∈I}$ be a simplicial partition of $\PNR$.
We say that the partition is \emph{adapted} to $p$ if it satisfies the conditions
of proposition \ref{adaptedeq}.
\end{defi}

\begin{coro}
  Let $p$ be an homogeneous polynomial of degree $d$ on
  $ℝ^{n+1}$. Let $(Δᵢ)_{i∈I}$ be a simplicial partition of $\PNR$ adapted to $p$.
  Then, the zero-locus of $p$ and $\overline{p}$ the $\Delta$ approximation of
  $p$ are isomorphic.
\end{coro}

This theorem is a consequence of the theorem \ref{maintheo} to come.

\begin{defi}
Let $p_1,\dots,p_k$ be $k$ homogeneous polynomial of degree $d_1,\dots,d_k$ on
$ℝ^{n+1}$. Let $(Δᵢ)_{i∈I}$ be a simplicial partition of $\PNR$.

We say that the partition is \emph{adapted} to $p_1,\dots,p_k$ if it satisfies
the following condition: for all $x \in ℝ^{n+1}$, either $\Pi_{p_j}(x)$ is true
for some $1 \leq j \leq k$ or $\ncone{p₁}{x} × \dots × \ncone{pₖ}{x}$ contains
only linear independant tuple of vectors.
\end{defi}

\begin{theo}\label{maintheo}
  Let $p_1,\dots,p_k$ be $k$ homogeneous polynomial of degree $d_1,\dots,d_k$ on
$ℝ^{n+1}$. Let $(Δᵢ)_{i∈I}$ be a simplicial partition of $\PNR$ adapted to
  $p₁,\dots,pₖ$. Then, the zero locus of $p₁,\dots,pₙ$ is isomorphic to the zero
  locus of $\overline{p}₁,\dots,\overline{p}ₖ$.
\end{theo}

We will prove our main theorem is the next two sections. The rest of the paper
explains how to test that a simplicial partition is adapted to a finite set of
polynomials.

%<!-- Local IspellDict: british -->
%<!-- Local IspellPersDict: ~/.ispell-british -->
