
We say that the partition is \emph{adapted} to $p$ if it satisfies the condition
of proposition \ref{adaptedeq}.
\begin{prop}
  Let $p$ be an homogeneous polynomial of degree $d$ on
$ℝ^{n+1}$.
  Let $(Δᵢ)_{i∈I}$ be a simplicial partition of $\PNR$ adapted to $p$.
  We can find a $C^∞$ function $δ$ in $ℝ^{n+1} ∖ \{0\} → ℝ^{n+1}∖\{0\} $
  and $μ > 0$ such that:
  $$∀x ∈  ℝ^{n+1}∖\{0\}, Π(x) \hbox{ or } \left\{
  \begin{array}{l}δ(x).∇p(x) > μ\|x\|^{d-1} \hbox{ and } \cr ∀(i,σ) ∈
  I(x), δ(x).∇ᵢ^σp > μ.\end{array}\right.
  $$
\end{prop}

\begin{proof}
From proposition \ref{adaptedeq}.\ref{hb}, for any $x ∈ ℝ^{n+1}∖\{0\}$, if
$Π_ε(x)$ is not satisfied, we find $δ₀(x)
∈ ℝ^{n+1}$ such that $δ₀(x).∇p(x) > 0$ and $∀(i,σ) ∈ I_ε(x), δ₀(x).∇ᵢ^σp > 0$.
We can also assume that $δ₀(λ x) = δ₀(x)$ for $λ > 0$ for the same reason as in
the proof of $(1) ↔ (3)$ above.

We start with the following lemma: for each $x ∈
ℝ^{n+1}∖\{0\}$, we find $0 < η(x) < ε$ such that
$∀y ∈ \ball{x}{2η(x)}$, $d₀(x).∇p(y) > 0$ and $∀(i,σ) ∈ I_ε(y),
d₀(x).∇ᵢ^σp > 0$.

First, $(i,σ) ∈ I_ε(y) ↔ \cball{y}{ε} ∩ Δᵢ^σ ≠ ∅$ is a closed condition and the
contrary is an open condition. The condition $Π_ε(x)$ is also an open
condition. Hence, for $x ∈ℝ^{n+1}∖\{0\}$, we find $η > 0$
small enough to have $y ∈
\ball{x}{2η}$ gives $Π_ε(x)$ or $(i,σ) ∉ I_ε(x)$ implies $(i,σ) ∉ I_ε(y)$.

which means that
for $y ∈ \ball{x}{2η}$, we have $Π_ε(x)$ or $I_ε(y) ⊆ I_ε(x)$ and therefore,

$$∀y ∈ \ball{x}{2η(x)}, Π_ε(x) \hbox{ or } ∀(i,σ) ∈ I_ε(y), δ₀(x).∇ᵢ^σp > 0.$$

Second, $∇p(x)$ is continuous, thus for  $x ∈ℝ^{n+1}∖\{0\}$, we find $η > 0$
small enough to have $y ∈ \ball{x}{2η}$
implies $Π_ε(x)$ or $δ₀(x).∇p(y) > 0$. Grouping both results yields the wanted real $η(x)$
such that

$$∀x ∈ℝ^{n+1}∖\{0\}, ∀y ∈ \ball{x}{2η(x)}, Π_ε(x) \hbox{ or }
\left\{\begin{array}{l} δ₀(x).∇p(y) > 0 \hbox{ and}\cr
∀(i,σ) ∈ I_ε(y), δ₀(x).∇ᵢ^σp > 0
\end{array}\right.
$$


Now, for $x ∈ \SNR$, the open balls $\ball{x}{η(x)}$ cover $\SNR$, so we can
find a finite family $(xⱼ)_{j∈J}$ such that the balls $\ball{x}{η(xⱼ)}$ cover
$\SNR$. We also take $η = \min_{j ∈ J} η(xⱼ)$.

Next, for each $xⱼ ∈ \SNR$, $j ∈ J$, we choose a measurable set
$Dⱼ ⊂ \ball{x}{η(xⱼ)}$ such that $\SNR$ is a disjoin union of all
$Dⱼ$. And we define $d₁ : \SNR → ℝ^{n+1}∖\{0\}$ with $δ₁(x) = δ₀(xⱼ)$ if $x ∈ Dⱼ$.

We choose $s : ℝ^{n+1} → ℝ$, a $C^∞$ function with support in
$\ball{0}{1}$ and $∫_{ℝ^{n+1}} s(t) dt = 1$ for the standard measure $dt$
on $ℝ^{n+1}$.

We define

$$ δ : \SNR → ℝ^{n+1}∖\{0\} \hbox{ with } δ(x) = ∫_{ℝ^{n+1}} δ₁(x - η t)
s(t) dt$$

Similarly to a convolution product, using the change of variable $u = x - ηt$, we have $δ(x) = ∫_{ℝ^{n+1}} η^{n +1} δ₁(u)
s(\frac{x - u}{η}) du$ and therefore $δ$ is $C^∞$ on $\SNR$.

Moreover, if we take $x ∈ \SNR$, we can write
$$ δ(x) = Σ_{j∈J} δ₁(xⱼ) ∫_{x - η t ∈ Dⱼ} s(t) dt$$

Let us define $λⱼ(x) = ∫_{x - η t ∈ Dⱼ} s(t) dt$. As $Dⱼ ⊂ \ball{xⱼ}{η(xⱼ)}$,
if $λⱼ(x) > 0$, we have $d(x - η t,xⱼ) ≤ η(xⱼ)$ otherwise $x  - η t ∉ Dⱼ$
and $d(x, x - η t) < η ≤ η(xⱼ)$ otherwise $s(t) = 0$. Therefore,
if $λⱼ(x) > 0$, we have $d(x,xⱼ) < 2η(xⱼ)$.
which implies $Π_ε(x)$ or both $δ₁(xⱼ).∇p(x) > 0$ and $∀(i,σ) ∈ I_ε(x), δ₁(xⱼ).∇ᵢ^σp > 0$.

Moreover, the coefficient $λⱼ$ are positive. Therefore, $δ(x) = Σ_{j∈J}
λⱼ(x) δ₁(xⱼ)$ implies $δ(x).∇p(x) > 0$ and $∀(i,σ) ∈ I_ε(x), δ(x).∇ᵢ^σp >
0$.  Thus, we have:
$$∀x ∈ \SNR, Π_ε(x) \hbox{ or } δ(x).∇p(x) > 0 \hbox{ and }∀(i,σ) ∈
  I_ε(x), δ(x).∇ᵢ^σp > 0.$$

  As $\SNR$ is compact, we can find $μ > 0$ such that
$$∀x ∈ \SNR, Π_ε(x) \hbox{ or } δ(x).∇p(x) > μ \hbox{ and }∀(i,σ) ∈
  I_ε(x), δ(x).∇ᵢ^σp > μ.$$


  To extend this to $ℝ^{n+1}∖\{0\}$, we just have to set $δ(λx) = δ(x)$ for $λ >
  0$ and as $p$ is homogeneous of degree $d$, we have:


  $$∀x ∈  ℝ^{n+1} ∖ \{0\}, Π_ε(x) \hbox{ or } \left\{\begin{array}{l}
  δ(x).∇p(x) > μ\|x\|^{d-1} \hbox{ and }\cr∀(i,σ) ∈
  I_ε(x), δ(x).∇ᵢ^σp > μ
  \end{array}\right.$$
\end{proof}

\begin{prop}
  Let $p$ be an homogeneous polynomial of degree $d$ on
$ℝ^{n+1}$.
  Let $(Δᵢ)_{i∈I}$ be an $ε$-fat simplicial partition of $\PNR$ adapted to $p$.
  the function  $p$ and $\overline{p}$ have homeomorphic zero locus on
  $ℝ^{n+1}∖\{0\}$.
\end{prop}

\begin{proof}
  The function $\overline{p}$ as a gradient defined in any direction $d$ which
  may be defined as
  $$
  \nabla \overline{p}(x). d = \lim_{h → 0⁺} \frac{\overline{p}(x + hd) - \overline{p}(x)}{h}
  $$
  This is true because for $h > 0$ and small enough, $x + hd$ and $x$ belong to
  at least one common cone $Δᵢ^σ$ with $(i, σ) ∈ I(x)$.
  If they both belong to more than one cone, as we take the gradient of linear
  functions that coincide on the intersection of several simplex in a direction which
  is parallel to this intersection, the gradient $∇ᵢ^σp . d$ coincide and are the limit.
  This tells us moreover that we have
  $$\nabla \overline{p}(x). d = ∇ᵢ^σp . d \hbox{ for some }
  (i, σ) ∈ I(x).$$

  Next, we Consider the differential equation $x'(t) = δ(x(t))$.
  This differential equation has
  unique solutions in the neighbourhood of $x$ because $δ$ is $C^∞$.

  Now consider a closed and connnected region $Ω$ of $ℝ^{n+1} ∖ \{0\}$
  where $p(x) ≤ 0$ and $\overline{p}(x) ≥ 0$, and a maximal solution
  $x : L → Ω$ with $0$ in the interior of the interval $L$ and $x(0)$ in
  the interior of $Ω$. The same reasoning will apply in region where $p(x) ≥ 0$
  and $\overline{p}(x) ≤ 0$.

  We define $\|p\| = max_{x ∈ \SNR} |p(x)|$ and  $\|\overline{p}\| = max_{x ∈
    \SNR} |\overline{p}(x)|$.

  From the definition of this region,
  no $x ∈ Ω$ satisfies the condition $Π_ε(x)$. Hence we have
  $d(x). ∇p(x) > μ\|x\|^{d-1} > 0$, so the solution is increasing in $p$.
  Similarly, $∀(i,σ) ∈ I_ε(x), d(x). ∇ᵢ^σ > μ > 0$, hence the solution is also
  increasing in $\overline{p}$.

  Therefore, for $t < 0$, we have $|p(x(0))| < |p(x(t))| < \|p\| \|x(t)\|ᵈ$.
  for $t > 0$, we have  $|\overline{p}(x(0))| < |\overline{p}(x(t))| <
  \|\overline{p}\| \|x(t)\|$.
  Let use take $R = min(\frac{\overline{p}(x(0))}{\|\overline{p}\|},
  \left(\frac{p(x(0))}{\|p\|}\right)^{\frac{1}{d}})$,
  we have $\|x(t)\| > R$.   This exludes solution converging to $0$.

  Next, consider $f(t) = p(x(t))$, $f'(t) = ∇p(x(t)).δ(x(t)) > μ R^{d-1} $
  Therefore $f(t) - f(0) >  ∫₀ᵗ f'(t) dt > t μ R^{d-1}$. Therefore $p(x(t)$
  reaches the zero locus of $p$ for $0 < t < - f(0) / μ R^{d-1}$.
  It will also reach  the zero locus of $\overline{p}$ for $0 > t > - f(0) / μ$.
  This excludes all solution with unbounded interval definition.

  From this we can define the homeomorphism $ϕ$ between the zero loci of $p$ and
  and $\overline{p}$. Consider a point $x₀$ such that $p(x₀) = 0$. If
  $\overline{p}(x₀) = 0$, we take $ϕ(x₀) = x₀$. If $\overline{p}(x₀) < 0$, we
  consider the solution of the equation $x'(t) = δ(x(t))$, with $x(0) = x₀$.
  This solution reaches $\overline{p}(x(t))$ for some $t₁ < 0$ and we take
  $ϕ(x₀) = x(t₁)$. If $\overline{p}(x₀) > 0$, we
  consider the solution of the equation $x'(t) = δ(x(t))$, with $x(0) = x₀$.
  This solution reaches $\overline{p}(x(t))$ for some $t₁ > 0$ and we take
  $ϕ(x₀) = x(t₁)$.

  The function $ϕ$ is bijective, because its inverse can be constructed in the
  same way and solution of the differential equation is unique. And it is
  $C^∞$ because $δ$ is $C^∞$.
\end{proof}



$$f_{ε,u,v}(x) = v p(x) + u \|x\|^{d-1} ∫ f(x - ε v \|x\| t) s(t) dt  $$

\begin{eqnarray*}
  ∇f_{ε,u,v}(x) &=& v ∇p(x) + u (d-1) x \|x\|^{d-3} ∫ f(x - ε v \|x\| t) s(t)
  dt \cr
  && + u \|x\|^{d-1} ∫ ∇f(x - ε v \|x\| t) s(t) dt  \cr
 && - ε v u \|x\|^{d-2} ∫ ∇f(x - ε v \|x\| t)^tx t s(t) dt
\end{eqnarray*}

If $∇f = 0$, then $f = 0$ and
$$∫ f(x - ε v \|x\| t) s(t) dt = - \frac{v p(x)}{u \|x\|^{d-1}} $$

If $∇f = 0$, then
\begin{eqnarray*}
  ∇f_{ε,u,v}(x) &=& v ∇p(x) + v (d-1) x \|x\|^{-2} p(x)
  dt \cr
  && + u \|x\|^{d-1} ∫ ∇f(x - ε v \|x\| t) s(t) dt  \cr
 && - ε v u \|x\|^{d-2} ∫ ∇f(x - ε v \|x\| t)^tx t s(t) dt
\end{eqnarray*}


\begin{defi}
   Let $Δ = (Δᵢ)_{i∈I}$ be  a \emph{simplicial partition} of $\PNR$, a
   \emph{$Δ$-linear map}
   is a map $f : ℝ^{n+1} → ℝ$ which is continuous, piecewise linear and linear
on each $Δᵢ⁺$ and $Δᵢ⁻$ for $i ∈ I$.

  For such a map, and for all $i∈I$, we define $fᵢ⁺$ (resp. $fᵢ⁻$) the restriction of $f$ to
  $Δᵢ⁺$ (resp. $Δᵢ⁻$). We also define $∇fᵢ⁺$ (resp. $∇fᵢ⁻$) the gradient of
  $fᵢ⁺$ (resp. $fᵢ⁻$), which is constant as these functions are linear.


  For any point $x∈\SNR$, we define $$∇ᶜf(x) = \cone{\{∇fᵢσ, (i,σ) ∈ I_ε(x ∈ Δᵢ⁺\} ∪ \{∇fᵢ⁻, x ∈ Δᵢ⁻\}}$$

  And for $x∈\SNR$ and $ε > 0$, we define
\begin{eqnarray*}
  ∇ᶜ_εf(x) &=& \cone{\{∇fᵢ⁺, \ball{x}{ε}
    ∩ Δᵢ⁺ ≠ ∅\} ∪ \{∇fᵢ⁻, \ball{x}{ε} ∩ Δᵢ⁻ ≠ ∅\}} \cr
  &=& \cone{\{∇f^σᵢ, (i, σ) ∈ I_ε(x)\}}
\end{eqnarray*}

\end{defi}

Remarks:
\begin{itemize}
  \item if $x ∈ πₙ^{-1}(\interior{Δᵢ})$
    for some $i ∈ I$, then $∇ᶜf(x) = \{∇f(x)\}$.

 \item In this paper, we will only consider odd or even $Δ$-linear map, meaning that
   it exists $σ ∈ \{-1,1\}$ such that for all $i ∈ I$ we have $fᵢ⁺ = σ fᵢ⁻$. For
   such a map, the zero locus is well defined on $\PNR$.
\end{itemize}

\begin{defi}
  We say that a $Δ$-linear map $f$ is smooth if for all $x ∈ ℝ^{n+1}∖\{0\}$, $∇ᶜf(x)$
  does not contains the vector $0$.
\end{defi}

\begin{prop}\label{ghull}
  If $Δ = (Δᵢ)_{i∈I}$ is an $ε$-fat \emph{simplicial partition} for some $ε>0$
  and if $f$ is a $Δ$-linear map, then for any $x ∈
     \SNR$ there exists $y ∈
     \SNR$ such that $$∇ᶜf(x) ⊆ ∇ᶜ_εf(x) ⊆ ∇ᶜf(y).$$ Moreover, if $f$ is smooth, then $0 ∉ ∇ᶜ_εf(x)$.
\end{prop}

\begin{proof}
  $∇ᶜf(x) ⊆ ∇ᶜ_εf(x)$ comes from $I(x) ⊂ I_ε(x)$.
  Let use define $J = I_ε(x)$. By definition $∇ᶜ_εf(x) = \cone{\{∇ᵢ^σ, (i,σ) ∈
    J\}}$.
  As the simplicial partition is $ε$-fat, we can take $y ∈ ∩_{(i,σ)
    ∈ J} Δᵢ^σ ∩ \SNR$. Therefore, $J ⊂ \{(i,σ), y ∈ Δᵢ^σ\}$ which implies
  $∇ᶜ_εf(x) ⊆ ∇ᶜf(y)$. The last part is a direct consequence of the definition
  of $f$ smooth.
\end{proof}

\begin{prop}
  Let $Δ = (Δᵢ)_{i∈I}$ be a \emph{simplicial partition} of $\PNR$ and
  let $f$ be a smooth \emph{$Δ$-linear map}.
  For any $ε_M ∈ ]0,1[$ such that $Δ$ is $ε_M$-fat, there exists a family of $C^∞$ function
  $ε ↦ f_ε : ]0,ε_M] → \SNR → ℝ$ such that
  \begin{enumerate}
  \item $∀x ∈ \SNR, f(x) - f_ε(x) → 0$ when $ε → 0$
  \item $∀x ∈ \SNR, ∇f_ε(x) ∈ ∇ᶜ_εf(x)$ and $∇f_ε(x) ≠ 0$.
  \item All functions $f_ε$ and $f$ have homeomorphic zero locus.
  \end{enumerate}
\end{prop}

\begin{proof} We first start with few notations:
  \begin{itemize}
  \item Let $s:ℝ^{n+1} → ℝ⁺$ a $C^∞$ map with support included in $B_1(0)$ and such
  that $∫_{ℝ^{n+1}} s(t)dt = 1$ for the usual measure $dt$ of $ℝ^{n+1}$. We also
  assume that $s(-x) = s(x)$.

  \item We define $s_ε(x) = ε^{-n-1}s(ε^{-1} x)$, which has support in $B_ε(0)$ and
  satisfies $∫_{ℝ^{n+1}} s_\epsilon(t)dt = 1$ because
  \begin{eqnarray*}
    ∫_{ℝ^{n+1}} s_\epsilon(t)dt &=& ∫_{ℝ^{n+1}} ε^{-n-1}s(ε^{-1}t)dt \cr
    &=& ∫_{ℝ^{n+1}} s(u)du \hbox{ using } t = εu, dt = ε^{n+1}du \cr
    &=& 1
  \end{eqnarray*}
  \end{itemize}


  We define $f_ε= f ∗ s_ε$: $$f_ε(x) = ∫_{ℝ^{n+1}} f(x-t)s_ε(t) dt = ∫_{ℝ^{n+1}}
  f(x-ε u)s(u) du.$$

  From the properties of convolution product we known that $f_ε$ is $C^∞$ on
  $\SNR$. For the limit:
  \begin{eqnarray*}
    |f(x) - f_ε(x)| &=& \left|∫_{ℝ^{n+1}} (f(x) - f(t))s_{ε}(x-t) dt\right|\cr
    &<& \max_{t∈B_ε(x)} |f(x) - f(t)| → 0  \hbox{ when } ε → 0\cr
    && \hbox{because $f$ is continuous.}
  \end{eqnarray*}

  We also define $\Psi_i^σ(x,ε) = \{ t ∈ \ball{0}{ε}, x - t ∈ Δᵢ^σ \}$

  For the gradient, using the fact that $f$ is $C¹$ almost every where and that
  $|∇f(x)|$ is bounded, we have
  \begin{eqnarray*}
    ∇f_ε(x) &=& ∇∫_{ℝ^{n+1}} f(x-t)s_{ε}(t) dt\cr
    &=& ∫_{ℝ^{n+1}} ∇f(x-t)s_{ε}(t) dt\cr
    &=& Σ_{(i,σ)∈I_ε(x)}∫_{\Psi_i^σ(x,ε)} ∇f(x-t)s_{ε}(t) dt\cr
    &=& Σ_{(i,σ)∈I_ε(x)} λᵢ(ε) ∇ᵢ^σf
    \hbox{ with } λᵢ(ε) = ∫_{\Psi_i^σ(x,ε)} s_{ε}(t) dt > 0
  \end{eqnarray*}


  Hence $∇f_ε(x)$ is in $∇ᶜ_εf(x)$ and hence non zero by proposition
  \ref{ghull} because
  $$Σ_{(i,σ)∈I_ε(x)} λᵢ(ε) = Σ_{i∈V_ε(x)}∫_{\Psi_i^σ(x,ε)} s_ε(t) dt =
  ∫_{\ball{0}{ε}} s_ε(t) dt = 1$$


  Moreover, we have
  \begin{eqnarray*}
    λᵢ(ε) &=&  ∫_{\Psi_i^σ(x,ε)} s_{ε}(t) dt \cr
    &=& ∫_{\Phi_i^σ(x,ε)} s(u) du
    \hbox{ with } \Phi_i^σ(x,ε) = \{ u ∈ \ball{0}{1}, x - εu ∈ Δᵢ^σ\}
  \end{eqnarray*}

  Now, if $ε$ is small enough to have $I(x) = I_ε(x)$, then $\Phi_i^σ(x,ε)$ also
  becomes independant of $ε$.  As a consequence, $∇f_ε(x)$ is constant for $ε$
  small enough, hence convergent to a value which is in $∇ᶜf(x)$ and therefore
  non zero. Therefore,
  we define $∇₀f(x)$ the limit of $∇f_ε(x)$.

  To build the homeomorphism between the various zero levels, we consider the
  following differential equation on for $γ(ε) : ℝ ↦ ℝ^{n+1}$:

  $$ γ'(ε) = g(ε,γ(ε)) \hbox{ with } g(ε,x) = - \frac{∂f_ε}{∂ε}(x) \frac{∇f_ε(x)}{||∇f_ε(x)||²} $$

  The function $g$ is ${\cal
    C}^∞$ for $0 < ε < ε_M$ and $x ∈ \SNR$ because $∇f_ε(x) ≠ 0$.
   We need to show that it can be
  prolongated to $ε = 0$. As we already have seen how to prolongate $∇f_ε(x)$ to
  a non zero value, we only have to examine the limit of $\frac{∂f_ε(x)}{∂ε}$.


  \begin{eqnarray*}
    \frac{∂f_ε(x)}{∂ε} &=& \frac{∂}{∂ε} ∫_{ℝ^{n+1}} f(x - ε u)s(u) du \cr
        &=& - ∫_{ℝ^{n+1}} ∇f(x - ε u).u s(u) du \cr
    &=& - Σ_{(i,σ)∈I_ε(x)} ∇ᵢ^σf. ∫_{\Phiᵢ^σ(x,ε)}  u s(u) du\cr
  \end{eqnarray*}

  For that same reason as above, the vector $∫_{\Phiᵢ^σ(x,ε)}  u s(u) du$
  is constant for   $ε$ is small enough. Hence $\frac{∂f_ε(x)}{∂ε}$ has a limit for $ε → 0$.

   Hence, by Cauchy-Lipschitz theorem, the above differential equation has
   a unique solution with $γ_{x₀}(ε) = x₀$ for $ε in [0,ε_M]$. Moreover $f_ε(γ(ε))$ is constant
   because
   \begin{eqnarray*}
    \frac{∂f_ε(γ_{x₀}(ε)}{∂ε} &=& \frac{∂}{∂ε} f_ε(γ_{x₀}(ε)) +
    ∇f_ε(γ_{x₀}(ε)).g(ε,γ_{x₀}(ε)) \cr
    &=& 0
   \end{eqnarray*}

   This means that $γ_{x₀}$ gives an homeomorphism between the zero locus of $f$
   and $f_ε$.
\end{proof}

\section{Polynomials and $Δ$-linear map}

  \begin{defi}
    Let $P$ be an homogeneous polynomial in $n+1$ variables.
    Let $(Δᵢ)_{i∈I}$ be a simplicial partition of $\PNR$. We define $f_P$ the
    piecewise linear map that coincides with $P$ on every summit of $Δ$.

    $(Δᵢ)_{i∈I}$ is $P$-thin if for any $i₁,\dots,iₖ ∈ I$ such that $Δ_{i₁} ∩
    \dots ∩ Δ_{iₖ} ≠ ∅$, the convex hull of
    $$ G_{i₁,\dots,iₖ} = \{ ∇ⱼ f_P, j = i₁,\dots,iₖ \} ∪ \{ ∇P(x), x ∈ Δ_{i₁} ∩
    \dots ∩ Δ_{iₖ} \}$$
    does not contain the vector $0$.
  \end{defi}


  \begin{prop}
    Let $P$ be an homogeneous polynomial in $n+1$ variables.
    Let $(Δᵢ)_{i∈I}$ be a $P$-thin simplicial partition of $\PNR$. Then the zero
    locus of $f_P$ and $P$ are homeomorphic.
  \end{prop}

  \begin{proof}
    Let $ε_M$ such that $(Δᵢ)_{i∈I}$ is $ε_M$ fat. For $(t,ε) ∈ [0,1] ×
    [0,ε_M]$, let $f_{t,ε} = (1-t) P + t f_ε$.
  \end{proof}
